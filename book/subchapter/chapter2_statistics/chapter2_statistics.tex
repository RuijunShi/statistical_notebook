\chapter{数理统计基础}
\section{概率}

1. 概率的定义(略)概率满足:非负性,规范性,可列可加性

2. 概率的性质:

   ​	重点:逆事件概率;加法公式;有限可加性

3. 条件概率:
\begin{equation}
    P(A|B)=\frac{P(AB)}{P(B)}
\end{equation}

4. 乘法定理:
\begin{equation}
    P(AB)=P(A|B)P(A)
\end{equation}

5. 全概率公式:
\begin{equation}
    P(A)=P\left(A \mid B_{1}\right) P\left(B_{1}\right)+
    P\left(A \mid B_{2}\right) P\left(B_{2}\right)+\ldots+
    P\left(A \mid B_{n}\right) P\left(B_{n}\right)
\end{equation}

6. 独立性:满足
\begin{subequations}
    \begin{align}
        P(AB)&=P(A)P(B) \\ 
        P(B|A)&=P(B)\\
    \end{align}
\end{subequations}

\section{单变量分布和多变量分布}
\setlength{\parindent}{2em}1. 随机变量的概念(略)

\indent2. 分布函数的概念(略)和性质:不减函数;$0\leq F(x)\leq 1 $; $F(x+0)=F(x)$
 
\indent3. 概率密度函数
 \begin{equation}
    F(x)=\int_{-\infty}^{x}f(t) \rm dt
\end{equation}
 
\begin{prop}{概率分布的性质}
    \begin{subequations}
    \begin{align}
        f(x)&\leq 0 \\ 
        \int_{-\infty}^{\infty}f(x)dx&=1\\
        P\{x_1<X<x_2\}&=\int_{x_1}^{x_2}f(x)dx\\
        F'(x)&=f(x)
    \end{align}
    \end{subequations}
\end{prop}


\section{随机变量的数学特征}
\subsection{数学期望与方差}

\begin{definition}{数学期望}
    \\积分:
    \begin{equation}
        E(X)=\int_{-\infty}^{\infty}xf(x)\mathrm{d} x
    \end{equation}
    为连续性随机变量的数学期望,离散状态下为:
    \begin{equation}
        E(X)=\sum_{k=1}^{\infty}x_k p_k
    \end{equation}
\end{definition}

\begin{prop}{方差的性质}
    \begin{itemize}
        \item 设$C$为常数,则有$E(C)=C$
        \item 设$C$为常数,$X$为随机变量,则有$$ E(CX)=CE(X) $$
        \item 设$X,Y$两个随机变量,则有:$$ E(X+Y)=E(X)+E(Y) $$
        \item 设$X,Y$是两个相互独立的随机变量,则有:$$ E(XY)=E(X)+E(Y) $$
    \end{itemize}
\end{prop}

\begin{definition}{方差\\}
    \setlength{\parindent}{2em}设$X$是一个随机变量,若$E\{ [X-E(X)]^2 \}$存在,则称为$E\{ [X-E(X)]^2 \}$为随机变量$X$的方差,记为$D(X)$或者$\mathrm{Var}(X)$
\end{definition}
\setlength{\parindent}{2em}根据定义,我们把方差写为:
\begin{equation}
    D(X)=\int^{\infty}_{-\infty} [x-E(x)]^2 f(x) \mathrm{d} x
\end{equation}
随机变量的方差可以写为:
\begin{equation}
    D(X)=E(X^2)-[E(X)]^2
\end{equation}

\begin{prop}{方差的性质}
    \begin{itemize}
        \item 设C为常数:D(C)=0
        \item 设C为常数,X为随机变量,有:$$D(CX)=C^2D(X),\qquad D(X+C)=D(X)$$
        \item 设X,Y为两个随机变量,有:$$D(X+Y)=D(X)+D(Y)+2E\{ (X-E(X))(Y-E(Y)) \}$$ 若X,Y相互独立,则有:$$ D(X+Y)=D(X)+D(Y) $$
        \item $D(X)=0$的充要条件是X以概率为1取常数$E(X)$,即:$$ P\{ X=E(X) \}=1 $$
    \end{itemize}
\end{prop}

\subsection{矩和协方差矩阵}
\begin{definition}{协方差}
    随机变量$E\{ (X-E(X))(Y-E(Y)) \}$称为变量X,Y的协方差,记为$\mathrm{Cov}(X,Y)$:
    \begin{equation}
        \mathrm{Cov}(X,Y)=E\{[X-E(X)][Y-E(Y)]\}
    \end{equation}
\end{definition}
\setlength{\parindent}{2em} 协方差是变量误差的一种描述。若随机变量X,Y完全独立则有$\mathrm{Cov}(X,Y)=0$
\begin{definition}{相关系数}
    \begin{equation}
        \rho_{X,Y}=\frac{\mathrm{Cov}(X,Y)}{\sqrt{D(X)}\sqrt{D(Y)}}
    \end{equation}
\end{definition}
当二维随机变量的二阶中心矩存在:
\begin{equation}
    \begin{aligned}
        c_{11}&=E\{[ X_1-E(X_1) ]^2\}\\
        c_{12}&=E\{[ X_1-E(X_1) ][ X_2-E(X_2) ]\}\\
        c_{21}&=E\{[ X_2-E(X_2) ][ X_1-E(X_1) ]\}\\
        c_{22}&=E\{[ X_2-E(X_2) ]^2\}
    \end{aligned}
\end{equation}
则矩阵$$ \left[\begin{matrix}
    c_{11}&c_{12}\\
    c_{21}&c_{22}
\end{matrix}\right] $$
称为协方差矩阵。若有$n$维随机变量,则矩阵:
\begin{equation}
    \mathbf{C}=\left[
    \begin{matrix}
        c_{11} & c_{12} & \cdots & c_{1n}\\
        c_{21} & c_{22} & \cdots & c_{2n}\\
        \vdots & \vdots & \ddots & \vdots\\
        c_{n1} & c_{n2} & \cdots & c_{nn}\\
    \end{matrix}
    \right]
\end{equation}
该矩阵是一个对称矩阵。
\subsection{多元正态分布及协方差矩阵的直观理解}
协方差矩阵描述随机变量的总体误差,而方差是协方差的一种特殊形式。协方差可以用多元正态分布直观理解其意义。
\section{常见分布及其期望方差}
\subsection{离散型分布}
1. (0-1)分布:
\begin{equation}
    P(X=k)=p^{k}(1-p)^{1-k}, 0<p<1, k=0,1
\end{equation}

2. 二项分布: 
\begin{equation}
    P(X=k)=\left(\begin{array}{l}n \\k\end{array}\right) p^{k}(1-p)^{n-k}
\end{equation}

3. 泊松分布: 
\begin{equation}
    P(X=k)=\frac{\lambda^{k} e^{-\lambda}}{k !}, k=0,1,2, \ldots
\end{equation}
\subsection{连续型分布}
4. Beta分布: 
   \begin{equation}
    f(x)=\frac{\Gamma(\alpha+\beta)}{\Gamma(\alpha) \Gamma(\beta)} x^{\alpha-1}(1-x)^{\beta-1}
   \end{equation}
   
   其中:$0\leq x\leq 1, \ \alpha>0,\ \beta>0,\ \Gamma(z)=\int_{0}^{+\infty} t^{z-1} e^{-t} d t$

5. 均匀分布: 
   \begin{equation}
        f(x)=\left\{\begin{array}{ll}\frac{1}{b-a} & a<x<b \\0 & \text { otherwise }\end{array}\right.
   \end{equation}
   
6. 指数分布: 
\begin{equation}
    f(x)=\left\{\begin{array}{ll}\frac{1}{\theta} e^{-x / \theta} & x>0 \\0 & \text { otherwise }\end{array}\right.
\end{equation}

7. 正态分布: 
\begin{equation}
    f(x)=\frac{1}{\sqrt{2 \pi} \sigma} e^{-\frac{(x-\mu)^{2}}{2 \sigma^{2}}}
\end{equation}
   $f(x)$关于$\mu$对称;$f(\mu)=max(f(x))=\frac{1}{\sqrt{2\pi}\sigma}$。

8. $Gamma$分布: 
\begin{equation}
   f(x)=\frac{\beta^{\alpha}}{\Gamma(\alpha)} x^{\alpha-1} e^{-\beta x}
\end{equation}
   其中:$x>0,\ \alpha>0,\ \beta>0$

9. Inv-$Gamma$分布: 
\begin{equation}
    f(x)=\frac{\beta^{\alpha}}{\Gamma(\alpha)} x^{-(\alpha+1)} e^{-\frac{\beta}{x}}
\end{equation}
   其中:$x>0,\ \alpha>0,\ \beta>0$

10. $\chi ^2$分布: 
\begin{equation}
    f_{k}(x)=\frac{1}{2^{\frac{k}{2}} \Gamma\left(\frac{k}{2}\right)} x^{\frac{k}{2}-1} e^{-\frac{x}{2}}
\end{equation}
    等价$\alpha=k/2,\beta = 1/2$的Gamma分布

11. Inv-$\chi ^2$分布: 
\begin{equation}
    f(x)=\frac{2^{-\frac{k}{2}}}{\Gamma\left(\frac{k}{2}\right)} x^{-\left(\frac{k}{2}+1\right)} e^{-\frac{1}{2 x}}
\end{equation}
    
    等价$\alpha=k/2,\beta = 1/2$的Inv-Gamma分布

12. Scaled Inv-$\chi ^2$分布: 
   \begin{equation}
       f(x)=\frac{\frac{k}{2}^{-\frac{k}{2}} s^{k}}{\Gamma\left(\frac{k}{2}\right)} x^{-\left(\frac{k}{2}+1\right)} e^{-\frac{k s^{2}}{2 x}}
   \end{equation}
     等价$\alpha=k/2,\beta = ks^2/2$的Inv-Gamma分布。

\section{大数定律与数理统计}


\section{误差传递}

\section{参数估计}

\section{假设检验}