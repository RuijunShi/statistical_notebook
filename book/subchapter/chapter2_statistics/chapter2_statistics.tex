\chapter{数理统计基础}
\section{概率}

1. 概率的定义(略)概率满足:非负性,规范性,可列可加性

2. 概率的性质:

   ​	重点:逆事件概率;加法公式;有限可加性

3. 条件概率:
\begin{equation}
    P(A|B)=\frac{P(AB)}{P(B)}
\end{equation}

4. 乘法定理:
\begin{equation}
    P(AB)=P(A|B)P(A)
\end{equation}

5. 全概率公式:
\begin{equation}
    P(A)=P\left(A \mid B_{1}\right) P\left(B_{1}\right)+
    P\left(A \mid B_{2}\right) P\left(B_{2}\right)+\ldots+
    P\left(A \mid B_{n}\right) P\left(B_{n}\right)
\end{equation}

6. 独立性:满足
\begin{subequations}
    \begin{align}
        P(AB)&=P(A)P(B) \\ 
        P(B|A)&=P(B)
    \end{align}
\end{subequations}

\section{单变量分布和多变量分布}
\subsection{单变量分布}
\setlength{\parindent}{2em}
1. 随机变量的概念(略)

\indent2. 分布函数的概念(略)和性质:不减函数;$0\leq F(x)\leq 1 $; $F(x+0)=F(x)$
 
\indent3. 概率密度函数
 \begin{equation}
    F(x)=\int_{-\infty}^{x}f(t) \rm dt
\end{equation}
 
\begin{prop}{概率分布的性质}
    \begin{subequations}
    \begin{align}
        f(x)&\leq 0 \\ 
        \int_{-\infty}^{\infty}f(x)dx&=1\\
        P\{x_1<X<x_2\}&=\int_{x_1}^{x_2}f(x)dx\\
        F'(x)&=f(x)
    \end{align}
    \end{subequations}
\end{prop}
\subsection{多变量分布}
\begin{definition}{二维随机变量\\}
    设$(X,Y)$是二维随机变量,对于任意实数$(x,y)$,二元函数$$F(x,y)=P\{ X\leq x, Y\leq y \}$$为二维随机变量$(X,Y)$的\textbf{分布函数}。在$(X,Y)$为离散型,离散状态下,二维随机变量$(x,y)$取值$(x_i,y_j)$, $i,j=1,2,\cdots$,则有:
    $$P\{x=x_i,Y=y_j\}=p_{ij},\qquad i,j=1,2,\cdots$$
    其中$p_{ij}\leq 0$且$\sum_i \sum_j p_{ij}=1$,则称为离散型随机随机变量的\textbf{分布律}或\textbf{联合分布律}
\end{definition}

\begin{definition}{联合概率密度\\}
    对于非负可积的函数$f(x,y)$,对于任意实数$x,y$有
    \begin{equation}
        F(X,Y)=\int_{-\infty}^{y}\int_{-\infty}^{x} f(u,v)\mathrm{d}x\mathrm{d}y
    \end{equation}
    则称为函数$f(x,y)$为$(X,Y)$的\textbf{概率密度}或\textbf{联合概率密度}
\end{definition}
\subsection{边缘分布}
\begin{definition}{\textbf{边缘分布}}
    $F(x,y)$是随机变量$(X,Y)$的分布函数,$F_X(x)$和$F_Y(y)$分别是X和Y的分布函数:
    \begin{equation}
        F_X(x)=p\{X\leq x\}=p\{X\leq x,Y\leq \infty\}=F(x,+\infty)
    \end{equation}
    对于随机变量Y同理。分布$F_X(x)$和$F_Y(y)$称为$(X,Y)$关于X和Y的\textbf{边缘分布函数}。
\end{definition}


对于离散型随机变量,边缘分布为:
\begin{equation}
    \begin{aligned}
        P\{X=x_i\}&=\sum_j p_{ij}=p_{\cdot i}\\
        P\{Y=y_i\}&=\sum_i p_{ij}=p_{\cdot j}
    \end{aligned}
\end{equation}
对于连续性随机变量,边缘概率密度为:
\begin{equation}
    f_X(x)=\int_{-\infty}^{+\infty}f(x,y)dy
    f_Y(y)=\int_{-\infty}^{+\infty}f(x,y)dx
\end{equation}
同理,我们可以扩展到n维分布中,假设要求维度a的边缘分布,则对其他维度$^- a$即可
\subsection{条件分布}
\begin{definition}{条件分布\\}
    对于二维随机变量$(X,Y)$,其分布律为:
    \begin{equation*}
        P\{X=x_i,Y=y_j\}=p_{ij},\qquad i,j=1,2,\cdots
    \end{equation*}
    对于固定的j,若$P\{Y=y_j\}>0$,则称为:
    \begin{equation}
        P\{X=x_i|Y=y_j\}=\frac{P\{X=x_i,Y=y_j\}}{P{Y=y_j}}=\frac{p_{ij}}{p_{i\cdot}}
    \end{equation}
    称为在$Y=y_i$条件下X的\textbf{条件分布率}
\end{definition}


\section{随机变量的数学特征}
\subsection{数学期望与方差}

\begin{definition}{数学期望}
    \\积分:
    \begin{equation}
        E(X)=\int_{-\infty}^{\infty}xf(x)\mathrm{d} x
    \end{equation}
    为连续性随机变量的数学期望,离散状态下为:
    \begin{equation}
        E(X)=\sum_{k=1}^{\infty}x_k p_k
    \end{equation}
\end{definition}
\begin{definition}{方差\\}
    设$X$是一个随机变量,若$E\{ [X-E(X)]^2 \}$存在,则称为$E\{ [X-E(X)]^2 \}$为随机变量$X$的方差,记为$D(X)$或者$\mathrm{Var}(X)$
\end{definition}


\setlength{\parindent}{2em}我们可以看到方差描述的是一个随机变量的离散程度。根据定义,我们把方差写为:
\begin{equation}
    D(X)=\int^{\infty}_{-\infty} [x-E(x)]^2 f(x) \mathrm{d} x
\end{equation}
随机变量的方差可以写为:
\begin{equation}
    D(X)=E(X^2)-[E(X)]^2
\end{equation}

\begin{prop}{方差的性质}
    \begin{itemize}
        \item 设C为常数:D(C)=0
        \item 设C为常数,X为随机变量,有:$$D(CX)=C^2D(X),\qquad D(X+C)=D(X)$$
        \item 设X,Y为两个随机变量,有:$$D(X+Y)=D(X)+D(Y)+2E\{ (X-E(X))(Y-E(Y)) \}$$ 若X,Y相互独立,则有:$$ D(X+Y)=D(X)+D(Y) $$
        \item $D(X)=0$的充要条件是X以概率为1取常数$E(X)$,即:$$ P\{ X=E(X) \}=1 $$
    \end{itemize}
\end{prop}

\subsection{矩和协方差矩阵}
矩在物理学中有广泛的应用,比如我们熟悉的力矩,电荷的分布等。力矩描述了力在空间上的分布;质量函数描述了质量在空间的分布等等,可以用下面的公式表征:
$$
\mu_n = \int r^n \rho(r)\mathrm{d} r
$$
而统计学中的矩也是类似,表征了概率密度函数的形状。类似的,我们也可以在数学上定义矩的概念:
\begin{definition}{矩的概念\\}
    在数学上,矩是对函数的一种度量,是描述概率分布的一种方法。对于单变量分布,对于常数$c$的k阶矩,有:
    \begin{equation}
        \mu_k=\int (x-c)^k P(x)\mathrm{d}x
    \end{equation}
\end{definition}
当$c=0,k=1$时,我们发现正是随机变量$X$的数学期望:
$$ \mu_1=\int xP(x)\mathrm{d}x = E(x) $$我们把$c=0,k=k$的情况称为k阶原点矩。若$c=E(X),k=k$,则称为k阶中心距。我们看到方差公式正是我们的二阶中心矩。当我们有2个变量的时候可以定义混合矩的概念:
$$\mu_{kl}=\iint (x-c_x)^k (y-c_y)^k P(x,y)\mathrm{d}x\mathrm{d}y$$
同样的,当$c_x,c_y=0$时,称为$k+l$阶混合矩:$E\{X^kY^l\}$;当$c_x=E(X),c_y=E(Y)$时,称为$k+l$阶混合中心矩:$E\{X^kY^l\}$。而随机变量$X,Y$的二阶混合中心矩则为协方差。
\begin{definition}{协方差}
    随机变量$E\{ (X-E(X))(Y-E(Y)) \}$称为变量X,Y的协方差,记为$\mathrm{Cov}(X,Y)$:
    \begin{equation}
        \mathrm{Cov}(X,Y)=E\{[X-E(X)][Y-E(Y)]\}
    \end{equation}
\end{definition}

协方差是变量误差的一种描述,衡量两个随机变量的相似性;而相关系数描述的是随机变量的相关性。若随机变量X,Y完全独立则有$\mathrm{Cov}(X,Y)=0$。

\begin{definition}{相关系数}
    \begin{equation}
        \rho_{X,Y}=\frac{\mathrm{Cov}(X,Y)}{\sqrt{D(X)}\sqrt{D(Y)}}
    \end{equation}
\end{definition}

当二维随机变量的二阶中心矩存在:
\begin{equation}
    \begin{aligned}
        c_{11}&=E\{[ X_1-E(X_1) ]^2\}\\
        c_{12}&=E\{[ X_1-E(X_1) ][ X_2-E(X_2) ]\}\\
        c_{21}&=E\{[ X_2-E(X_2) ][ X_1-E(X_1) ]\}\\
        c_{22}&=E\{[ X_2-E(X_2) ]^2\}
    \end{aligned}
\end{equation}
则矩阵$$ \left[\begin{matrix}
    c_{11}&c_{12}\\
    c_{21}&c_{22}
\end{matrix}\right] $$
称为协方差矩阵。若有$n$维随机变量,
\begin{equation}
    c_{ij}=\mathrm{Cov}(X_i,X_j),i,j=1,2,\cdots,n
\end{equation}
则矩阵:
\begin{equation}
    \mathbf{C}=\left[
    \begin{matrix}
        c_{11} & c_{12} & \cdots & c_{1n}\\
        c_{21} & c_{22} & \cdots & c_{2n}\\
        \vdots & \vdots & \ddots & \vdots\\
        c_{n1} & c_{n2} & \cdots & c_{nn}\\
    \end{matrix}
    \right]
\end{equation}
该矩阵是一个对称矩阵。在对角线上则为该变量的方差。

\begin{definition}{矩母函数\\}
    定义矩母函数为:\begin{equation}
        \psi(t)=E[e^{tX}]=\int e^{tX}dF(x)
    \end{equation}
\end{definition}
对矩母函数求$n$次导可得:
\begin{equation}
    \psi^{n}(t) = E[X^n e^{tX}]
\end{equation}
当$t=0$时,我们发现:
\begin{equation}
    \psi ^n(0) = E[X^n]
\end{equation}
这正好对应了我们的n阶原点矩公式。
\subsection{多元正态分布及协方差矩阵的直观理解}
协方差矩阵描述随机变量的总体误差,表示随机变量之间的相似程度;而方差是协方差的一种特殊形式。协方差可以用多元正态分布直观理解其意义。对于一个边缘分布为正态分布的二维分布,其概率密度分布为:
\begin{equation}
    f(x_1,x_2)=\frac{1}{2\pi\sigma_1\sigma_2\sqrt{1-\rho^2}}\exp
    \left\{ -\frac{1}{2(1-\rho^2)}\left[\frac{(x_1-\mu_1)^2}{\sigma^2_1}-2\rho\frac{(x_1-\mu_1)(x_2-\mu_2)}{\sigma_1^2\sigma_2^2}+\frac{(x_2-\mu_2)^2)}{\sigma^2_2}\right] \right\}
\end{equation}
我们知道二维正态分布的协方差为:$\mathrm{Cov}(X,Y)=\rho \sigma_1\sigma_2$
其协方差矩阵为:
$$\mathbf{C}=\left[
    \begin{matrix}
        c_{11} & c_{12} \\
        c_{21} & c_{22}
        \end{matrix}\right]=\left[
    \begin{matrix}
        \sigma_1^2 & \rho\sigma_1\sigma_2 \\
        \rho\sigma_1\sigma_2 & \sigma_2^2
    \end{matrix}
\right]$$
记$$\mathbf{X}=\left[
    \begin{matrix}
        x_1 \\
        x_2
    \end{matrix}\right],\quad
    \mathbf{\mu}=\left[
        \begin{matrix}
            \mu_1 \\
            \mu_2
        \end{matrix}\right]$$
    
经过计算,我们发现:
$$\begin{aligned}
    &(\mathbf{X}-\mathbf{\mu})^T \mathrm{C}^{-1}(\mathbf{X}-\mathrm{\mu})\\
    &=\frac{1}{2(1-\rho^2)}\left[\frac{(x_1-\mu_1)^2}{\sigma^2_1}-2\rho\frac{(x_1-\mu_1)(x_2-\mu_2)}{\sigma_1^2\sigma_2^2}+\frac{(x_2-\mu_2)^2)}{\sigma^2_2}\right] 
\end{aligned}$$
因此二维正态分布可以写为:
$$f(x_1,x_2)=\frac{1}{(2\pi)^{2/2}(\det\mathbf{C})^{1/2}}\exp\left\{
    -\frac{1}{2}(\mathbf{X}-\boldsymbol{\mu})^T \mathbf{C}^{-1}(\mathbf{X}-\boldsymbol{\mu})
\right\}$$
我们发现N维正态分布是由协方差矩阵$\mathbf{C}$规定的。当协方差$\mathbf{C}$改变时,多维正态分布的函数形状也会依此改变。

------此处应该有图像-------

我们可以从二维正态分布扩展到n维正态分布:
\begin{equation}
    f(X)=\frac{1}{(2\pi)^{n/2} (\det \mathbf{C})^{1/2}}\exp
    \left\{
    -\frac{1}{2}(\mathbf{X}-\boldsymbol{\mu})^T \mathbf{C}^{-1}(\mathbf{X}-\boldsymbol{\mu})
    \right\}
\end{equation}
这就是N维正态分布的一般形式。n维正态分布的性质有:
\begin{prop}{\textbf{n维正态分布的性质}}
    \begin{itemize}
        \item 若$(X_1,X_2,\cdots,X_n)$服从n维正态分布,对于任意的$X_i(i=1,2,\cdots,n)$服从一维正态分布;若$X_1,X_2,\cdots,X_n$均服从正态分布且相互独立,则$(X_1,X_2,\cdots,X_n)$为服从n维正态分布。
        \item n正态维随机变量$(X_1,X_2,\cdots,X_n)$服从n维正态分布的充要条件是$X_1,X_2,\cdots,X_n$的线性组合:$$l_1X_1+l_2X_2+\cdots+l_nX_n$$
        \item 若$(X_1,X_2,\cdots,X_n)$服从n维正态分布,而$Y_1,Y_2,\cdots,Y_n$与$X_1,X_2,\cdots,X_n$是线性变化,则$(Y_1,Y_2,\cdots,Y_n)$也服从n维正态分布。
        \item 若$(X_1,X_2,\cdots,X_n)$服从n维正态分布,则$X_1,X_2,\cdots,X_n$线性无关。
    \end{itemize}
\end{prop}

\subsection{偏度和峰度}
对于概率分布的形状可以用偏度系数和峰度系数表征。首先我们引入标准矩的概念:
\begin{definition}\textbf{标准矩}
    一个概率分布的标准矩是经过标准化后的中心矩。标准化通常是将其除以标准差的过程,这样做可以使得标准矩对缩放和离散程度皆能保持一致, 在比较不同概率分布的形状时更为方便。
    \begin{equation}
        \hat{\mu}_k = \frac{\mu_k}{\sigma^k}=\frac{\mathrm{E}[(X-\mu)^k]}{(\mathrm{E}[(X-\mu)^2])^{k/2}}
    \end{equation}
\end{definition}

因为进行了标准化,因此标准矩具有缩放不变性。结合上一节对中心矩的定义,一阶标准矩即为随机变量标准化后的一阶中心距;因此一阶标准矩恒为0;二阶标准矩恒为1。而下面要定义的偏度则为三阶标准矩;峰度则为四阶标准矩。
\begin{definition}\textbf{偏度系数}
    表征分布形态与平均值偏离程度,作为分布不对称的测度:
    \begin{equation}
        g_1 = \hat{\mu}_3 = \frac{\mu_3}{\sigma^3}=\frac{\mathrm{E}[(X-\mu)^3]}{(\mathrm{E}[(X-\mu)^2])^{3/2}}
    \end{equation}
\end{definition}

\begin{definition}{\textbf{峰度系数}}
    表征分布形态图形顶峰的凸平度(即渐进于横轴的陡度):
    \begin{equation}
        g_2 = \hat{\mu}_4 = \frac{\mu_4}{\sigma^4}=\frac{\mathrm{E}[(X-\mu)^4]}{(\mathrm{E}[(X-\mu)^4])^{4/2}}
    \end{equation}
\end{definition}


\section{常见分布及其数学特征}
\subsection{离散型分布}

\begin{enumerate}
\item (0-1)分布:
\begin{equation}
    P(X=k)=p^{k}(1-p)^{1-k},\qquad 0<p<1, k=0,1
\end{equation}
0-1分布是最简单的分布,一个只有两种结果的随机现象即为0-1分布。其期望为$p$,方差为$p(1-p)$

\item 二项分布: 当有n次0-1分布时即为二项分布。
\begin{equation}
    P(X=k)=\left(\begin{array}{l}n \\k\end{array}\right) p^{k}(1-p)^{n-k}
\end{equation}
其中$\left(\begin{matrix} n\\k \end{matrix}\right)=\frac{n!}{k!(n-k)!}$为二项式系数。
均值:$np$; 方差:$np(1-p)$

\item 泊松分布: 
\begin{equation}
    P(X=k)=\frac{\lambda^{k} e^{-\lambda}}{k !}, k=0,1,2, \ldots
\end{equation}
当二项分布的n很大而p很小时候,泊松分布即为二项分布的近似。均值:$\lambda$; 方差: $\lambda$。这是最重要的离散分布。
\end{enumerate}
\subsection{连续型分布}
首先介绍下$\Gamma$ 函数:
\begin{equation}
    \Gamma(z)=\int_{0}^{+\infty} t^{z-1} e^{-t} \mathrm{d} t
\end{equation}
这个函数在下面会经常用到。
\begin{enumerate}
\item 均匀分布: 
   \begin{equation}
        f(x)=\left\{\begin{array}{ll}\frac{1}{b-a} & a<x<b \\0 & \text { otherwise }\end{array}\right.
   \end{equation}
   
\item 指数分布: 
\begin{equation}
    f(x)=\left\{\begin{array}{ll}\frac{1}{\theta} e^{-x / \theta} & x>0 \\0 & \text { otherwise }\end{array}\right.
\end{equation}
指数分布有一个非常重要的性质是无记忆性。

\item 正态分布: 
\begin{equation}
    f(x)=\frac{1}{\sqrt{2 \pi} \sigma} e^{-\frac{(x-\mu)^{2}}{2 \sigma^{2}}}
\end{equation}
$f(x)$关于$\mu$对称;$f(\mu)=\max[f(x)]=\frac{1}{\sqrt{2\pi}\sigma}$。均值为$\mu$;方差为$\sigma ^2$。当$\mu = 0$,$\sigma=1$时的正态分布称为标准正态分布;对于一个正态分布:$X\sim N(\mu,\sigma^2)$,则有变量:
$$Z=\frac{X-\mu}{\sigma}\sim N(0,1)
$$
变量Z称为标准化变量。正态分布又称为高斯分布,多维高斯分布在天文中也是非常重要的,很多参数的估计开始都要考虑一个多维正态分布。

\item  Beta分布: 
   \begin{equation}
    f(x)=\frac{\Gamma(\alpha+\beta)}{\Gamma(\alpha) \Gamma(\beta)} x^{\alpha-1}(1-x)^{\beta-1}
   \end{equation}
其中:$0\leq x\leq 1, \ \alpha>0,\ \beta>0,\ \Gamma(z)=\int_{0}^{+\infty} t^{z-1} e^{-t} \mathrm{d} t$。其期望为:$\frac{a}{a+b}$,方差为:$\frac{ab}{(a+b)^2(a+b+1)}$
Beta分布是从伯努利事件建模的出来的,Beta分布描述了一个事物出现的所有可能性的大小分布。

\item Gamma分布: 
\begin{equation}
   f(x)=\frac{\beta^{\alpha}}{\Gamma(\alpha)} x^{\alpha-1} e^{-\beta x}
\end{equation}
其中:$x>0,\ \alpha>0,\ \beta>0$; 数学期望为:$\frac{1}{\lambda}$; 方差为:$\frac{n}{\lambda ^2}$

\item Inv-Gamma分布: 
\begin{equation}
    f(x)=\frac{\beta^{\alpha}}{\Gamma(\alpha)} x^{-(\alpha+1)} e^{-\frac{\beta}{x}}
\end{equation}
其中:$x>0,\ \alpha>0,\ \beta>0$

\item $\chi ^2$分布: 
\begin{equation}
    f_{k}(x)=\frac{1}{2^{\frac{k}{2}} \Gamma\left(\frac{k}{2}\right)} x^{\frac{k}{2}-1} e^{-\frac{x}{2}}
\end{equation}
等价$\alpha=k/2,\beta = 1/2$的Gamma分布

\item  Inv-$\chi ^2$分布: 
\begin{equation}
    f(x)=\frac{2^{-\frac{k}{2}}}{\Gamma\left(\frac{k}{2}\right)} x^{-\left(\frac{k}{2}+1\right)} e^{-\frac{1}{2 x}}
\end{equation}
等价$\alpha=k/2,\beta = 1/2$的Inv-Gamma分布

\item  Scaled Inv-$\chi ^2$分布: 
   \begin{equation}
       f(x)=\frac{\frac{k}{2}^{-\frac{k}{2}} s^{k}}{\Gamma\left(\frac{k}{2}\right)} x^{-\left(\frac{k}{2}+1\right)} e^{-\frac{k s^{2}}{2 x}}
   \end{equation}
     等价$\alpha=k/2,\beta = ks^2/2$的Inv-Gamma分布。
\end{enumerate}
这些分布各有各的作用,也有其数学上和现实的意义。
\section{概率密度的传递}
如果我们需要从一个已知的变量x分布转换为另外变量y的分布(而一般这种变换是非线性的),这时候需要概率密度的传递公式。

对于随机变量X的概率密度函数$f_X(x)$已知的情况下,需要得到随机变量Y的概率密度函数,而随机变量X和Y有一个非线性变换的关系:$x=g(y)$;也就是说函数$g$是函数$f$的反函数。则概率密度函数$f(x)$转变为$\widetilde{f}(y)=f(g(y))$。这时候对于随机变量X和随机变量Y有关系:
\begin{equation}
    p_X(x)\delta x \simeq p_Y(y)\delta y
\end{equation}
因此对于随机变量Y的概率密度分布为:
\begin{equation}
    \begin{aligned}
        p_Y(y)&=p_X(x)\left| \frac{\textrm{d}x}{\textrm{d}y} \right|\\
        &=p_X[g(y)]|g'(y)|
    \end{aligned}
\end{equation}

若对随机变量X和随机变量Y不是一维变量,而是多维变量$X$和$Y$,其反函数$g(y)$与随机变量$X$的概率密度函数$f$一一对应,则有等式:
\begin{equation*}
    p_{X}(x_i) = \sum_i p_{Y}(y_i)
\end{equation*}
这时候我们的误差传递公式可以写为:
\begin{equation}
    p_\mathbf{Y}(y)=|\mathbf{J}|p_X[g(y)]
\end{equation}
其中$\mathbf{J}$为Jacobian矩阵:
\begin{equation}
    \mathbf{J}=
    \begin{bmatrix}
        \frac{\partial x_1}{\partial y_1} & \cdots & \frac{\partial x_1}{\partial y_n}\\
        \vdots & \ddots & \vdots \\
        \frac{\partial x_n}{\partial y_1} & \cdots  & \frac{\partial x_n}{\partial y_n}
    \end{bmatrix}
\end{equation}


\section{从大数定律,中心极限定理到数理统计}

\subsection{大数定理}
\begin{theorem}{弱大数定理,辛钦大数定理\\}
    设$X_1,X_2,\cdots$是独立同分布的,且数学期望为$E(x_k)=\mu$,前n个变量的算术平均为$\frac{1}{n}\sum_{k=1}^{n}X_K$,则对于任意的的$\varepsilon > 0$有:
    \begin{equation}
        \lim_{n\rightarrow \infty}P \left\{
        \left\lvert \frac{1}{n}\sum_{k=1}^{n}X_K - \nu \right \rvert < \varepsilon
        \right\}=1
    \end{equation}
\end{theorem}

\begin{theorem}{强大数定理\\}
    若$X_1,X_2,\cdots$独立同分布,且都有均值$\mu$,则:
    \begin{equation}
        P\{ \lim_{n\rightarrow \infty}(X_1+\cdots+X_n)/n=\mu \}=1
    \end{equation}
\end{theorem}


\subsection{中心极限定理}
\begin{theorem}{中心极限定理\\}
    若$X_1,X_2,\cdots$独立同分布,且都有均值$\mu$和方差$\sigma ^2$,则:
    \begin{equation}
        \lim_{n\rightarrow \infty} 
        P\left\{\frac{X_1+\cdots+X_n-n\mu}{\sigma \sqrt{n}}  \leq a \right\}=
        \int _{-\infty}^{a} \frac{1}{\sqrt{2\pi}}e^{-x^2/2}\mathrm{d}x
    \end{equation}
\end{theorem}


大数定理告诉我们只要采样n足够大就能逼近期待值;而中心极限定理说明了如果采样n逼近无穷,会呈现一个正态分布。这也是正态分布在统计学上非常重要的原因之一。大数定理和中心极限定理是概率论与数理统计的桥梁,当我们样本足够大的时候可以反映一个事物的统计性质,这也为下面从随机抽样到数理统计建设起一条桥梁。

\subsection{随机抽样}
我们在进行随机试验的时候,很多情况下可以用数字表示,或者是可以量化的结果。在随机试验中,我们把所有可能观察值称为\textbf{总体},而每一个观察值称为\textbf{个体};总体中包含的个体数目称为\textbf{容量}:根据容量的有限或者无限,又可以区分为\textbf{有限总体}和\textbf{无限总体}。如果我们在一个分布F中抽取,我们可以定义样本:
\begin{definition}{样本}
    设X是从分布函数F中抽取的随机变量。若$X_1,X_2,\cdots,X_n$是具有同一分布函数F(或概率密度f),相互独立的随机变量,则称为$X_1,X_2,\cdots,X_n$为u分布函数F得到的容量为n的简单随机样本,将成为样本。n个随机样本的值$x_1,x_2,\cdots,x_n$称为样本值。
\end{definition}
因此样本$X_1,X_2,\cdots,X_n$的概率密度为:
\begin{equation}
    f(x_1,x_2,\cdots,x_n)=\prod_{i=1}^n f(x_i)
\end{equation}

\section{参数估计}
\subsection{点估计}
\subsection{最大似然估计}
\subsection{区间估计}
\subsection{置信区间}

\section{假设检验}
假设检验的出发点是:小概率事件不可能发生。由于我们对一个总体数据没有办法进行完全抽样统计,只能抽出一部分样本来估计总体情况。因此假设检验就是提出一个\textbf{原假设$H_0$},然后对其假设进行统计推断,做出接受假设$H_0$还是拒绝假设$H_0$。首先假设$H_0$是正确的时,若抽样得到的样本d导致了小概率事件的发生,则拒绝原假设$H_0$,否之接受假设$H_0$。

当接受假设$H_0$时,则拒绝假设$H_1$;当拒绝假设$H_0$时,则为接受假设$H_1$。由于我们的假设不可能是完全正确的,这时候就会出现两类错误:第一类错误是\textbf{弃真错误:当假设$H_0$真时拒绝原假设};第二类错误是\textbf{取伪错误:当假设$H_0$假时接受原假设}。

\subsection{显著性检验}