\chapter{数理统计基础}
\section{概率}

1. 概率的定义(略)概率满足:非负性,规范性,可列可加性

2. 概率的性质:

   ​	重点:逆事件概率;加法公式;有限可加性

3. 条件概率:
\begin{equation}
    P(A|B)=\frac{P(AB)}{P(B)}
\end{equation}

4. 乘法定理:
\begin{equation}
    P(AB)=P(A|B)P(A)
\end{equation}

5. 全概率公式:
\begin{equation}
    P(A)=P\left(A \mid B_{1}\right) P\left(B_{1}\right)+
    P\left(A \mid B_{2}\right) P\left(B_{2}\right)+\ldots+
    P\left(A \mid B_{n}\right) P\left(B_{n}\right)
\end{equation}

6. 独立性:满足
\begin{subequations}
    \begin{align}
        P(AB)&=P(A)P(B) \\ 
        P(B|A)&=P(B)\\
    \end{align}
\end{subequations}

 \section{单变量分布}
 1. 随机变量的概念(略)

 2. 分布函数的概念(略)和性质:不减函数;$0\leq F(x)\leq 1 $; $F(x+0)=F(x)$
 
 3. 概率密度函数
 \begin{equation}
    F(x)=\int_{-\infty}^{x}f(t) \rm dt
\end{equation}
 
 性质:
\begin{subequations}
\begin{align}
    f(x)&\leq 0 \\ 
    \int_{-\infty}^{\infty}f(x)dx&=1\\
    P\{x_1<X<x_2\}&=\int_{x_1}^{x_2}f(x)dx\\
    F'(x)&=f(x)
\end{align}
\end{subequations}

\section{常见分布}
1. (0-1)分布:
\begin{equation}
    P(X=k)=p^{k}(1-p)^{1-k}, 0<p<1, k=0,1
\end{equation}

2. 二项分布: 
\begin{equation}
    P(X=k)=\left(\begin{array}{l}n \\k\end{array}\right) p^{k}(1-p)^{n-k}
\end{equation}

3. 泊松分布: 
\begin{equation}
    P(X=k)=\frac{\lambda^{k} e^{-\lambda}}{k !}, k=0,1,2, \ldots
\end{equation}

4. Beta分布: 
   \begin{equation}
    f(x)=\frac{\Gamma(\alpha+\beta)}{\Gamma(\alpha) \Gamma(\beta)} x^{\alpha-1}(1-x)^{\beta-1}
   \end{equation}
   
   其中:$0\leq x\leq 1, \ \alpha>0,\ \beta>0,\ \Gamma(z)=\int_{0}^{+\infty} t^{z-1} e^{-t} d t$

5. 均匀分布: 
   \begin{equation}
        f(x)=\left\{\begin{array}{ll}\frac{1}{b-a} & a<x<b \\0 & \text { otherwise }\end{array}\right.
   \end{equation}
   
6. 指数分布: 
\begin{equation}
    f(x)=\left\{\begin{array}{ll}\frac{1}{\theta} e^{-x / \theta} & x>0 \\0 & \text { otherwise }\end{array}\right.
\end{equation}

7. 正态分布: 
\begin{equation}
    f(x)=\frac{1}{\sqrt{2 \pi} \sigma} e^{-\frac{(x-\mu)^{2}}{2 \sigma^{2}}}
\end{equation}
   $f(x)$关于$\mu$对称;$f(\mu)=max(f(x))=\frac{1}{\sqrt{2\pi}\sigma}$。

8. $Gamma$分布: 
\begin{equation}
   f(x)=\frac{\beta^{\alpha}}{\Gamma(\alpha)} x^{\alpha-1} e^{-\beta x}
\end{equation}
   其中:$x>0,\ \alpha>0,\ \beta>0$

9. Inv-$Gamma$分布: 
\begin{equation}
    f(x)=\frac{\beta^{\alpha}}{\Gamma(\alpha)} x^{-(\alpha+1)} e^{-\frac{\beta}{x}}
\end{equation}
   其中:$x>0,\ \alpha>0,\ \beta>0$

10. $\chi ^2$分布: 
\begin{equation}
    f_{k}(x)=\frac{1}{2^{\frac{k}{2}} \Gamma\left(\frac{k}{2}\right)} x^{\frac{k}{2}-1} e^{-\frac{x}{2}}
\end{equation}
    等价$\alpha=k/2,\beta = 1/2$的Gamma分布

11. Inv-$\chi ^2$分布: 
\begin{equation}
    f(x)=\frac{2^{-\frac{k}{2}}}{\Gamma\left(\frac{k}{2}\right)} x^{-\left(\frac{k}{2}+1\right)} e^{-\frac{1}{2 x}}
\end{equation}
    
    等价$\alpha=k/2,\beta = 1/2$的Inv-Gamma分布

12. Scaled Inv-$\chi ^2$分布: 
   \begin{equation}
       f(x)=\frac{\frac{k}{2}^{-\frac{k}{2}} s^{k}}{\Gamma\left(\frac{k}{2}\right)} x^{-\left(\frac{k}{2}+1\right)} e^{-\frac{k s^{2}}{2 x}}
   \end{equation}
     等价$\alpha=k/2,\beta = ks^2/2$的Inv-Gamma分布。


\section{多元随机变量}

\section{随机变量的数学特征}

\section{频率学派的参数估计}

\section{频率学派的假设检验}