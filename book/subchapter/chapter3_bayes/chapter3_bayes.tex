\chapter{贝叶斯统计}
\section{贝叶斯定理推导}
1. 贝叶斯公式
\begin{equation}
    P(B_i|A)=\frac{P(A|B_i)P(B_i)}{\sum_{j=1}^{n}P(A|B_i)P(B_j)}
    =\frac{P(A|B_i)P(B_i)}{P(A)}
\end{equation}
先验:$P(B)$

似然:  $P(A|B)$

后验:  $P(B|A)$

证据(归一化):  $P(A)$

2. 贝叶斯公式含义:通过数据推算模型参数的概率。即:
\begin{equation}
    P({\rm{Model}} (\theta)|{\rm{Data}})=
    P({\rm{Data}}|{\rm{Model}}(\theta))P(\theta)
\end{equation}


3. 贝叶斯统计的优势:将这个某种程度上是主观性的信息明确表达在先验概率中,而不是隐藏在没有明确指出的假设中; 让数据说话,减少主观性的先验概率。


\section{贝叶斯学派和频率学派}



\section{单变量贝叶斯参数估计}

\section{多变量贝叶斯参数估计}

\section{分层贝叶斯模型}

\section{贝叶斯回归}

\section{贝叶斯模型选择}

