\chapter{贝叶斯统计}
\section{贝叶斯定理}
贝叶斯定理和贝叶斯参数估计是在计算天文学课上的笔记,缺少例子解释。这部分尽量找一些比较容易理解的例子。上来就直接讲贝叶斯定理和贝叶斯参数估计是很晦涩难懂的。

贝叶斯公式
首先我们回忆下条件概率的定义,条件B下A的概率为:$$ P(A|B)=\frac{P(AB)}{P(B)} $$我们反过来,在A条件下B的概率为:$$P(B|A)=\frac{P(AB)}{P(A)}$$
我们发现A,B的联合概率分布是相等量。因此移项可得:
$$P(A|B)P(B)=P(B|A)P(A)$$
如果我们很容易得到B条件下A的概率,那么我们要算A条件下B的概率就可以用上面的公式:
$$P(B|A)=\frac{P(A|B)P(B)}{P(A)}$$
若有n个样本,则贝叶斯公式为:
\begin{equation}
    P(B_i|A)=\frac{P(A|B_i)P(B_i)}{\sum_{j=1}^{n}P(A|B_i)P(B_j)}
    =\frac{P(A|B_i)P(B_i)}{P(A)}
\end{equation}

先验:$P(B)$

似然:  $P(A|B)$

后验:  $P(B|A)$

证据(归一化):  $P(A)$

这个公式是不是很难懂。在这里举个例子:有三个门,有一头牛两头羊。小明想养牛耕地。小明选择了其中一扇门,主持人会打开一个一定是羊的门,这时候小明可以改变自己的选择,玩家要不要改变自己的策略呢?

当第一次选择时候,三个门里面有牛的概率均为1/3,因此不换门获得牛的概率为1/3;开门后我后面得到牛的概率不是1/2吗?等等,这其实有点不对,因为主持人知道了三个门分别是什么。由于主持人一定会打开羊的门(主持人知道),如果换门有三种情况:
\begin{itemize}
    \item 选择的是羊,换门得到牛
    \item 选择的是羊,换门得到牛
    \item 选择的是牛,换门得到羊
\end{itemize}
那这么算下来换门的获胜的概率是2/3!我们从贝叶斯的角度上看,当主持人没有开门的时候,假设我获得牛的概率为$P(cowA)=1/3$; 当主持人打开门的时候,假设牛在A门,那主持人一定会选择B或C门打开,这时候主持人打开B门的概率为:$P(openB|cowA)=1/2$;如果牛在C门后,那么主持人只会选择把B打开,$P(OpenB|cowC)=1$;这时候我们可以计算主持人打开门B之后牛在A门的后验:
$$ p(cowC|openB) = \frac{p(openB|cowA)p(cowA)}{p(openB|cowA)p(cowA)+p(openB|cowC)p(cowC)}=1/3 $$
由于牛在A门和C门属于互斥事件,因此在打开门B之后牛出现在C门的概率为2/3.因此换门是更好的选择。从感性上理解,由于主持人打开了B门,也就是假设出现在A门(我们选的门)时候的的概率p(cowA)是1/2;而假设牛出现的C门的概率能达到1,因此有没有牛的先验概率已经从主持人的行为中“学习”到了。贝叶斯定理的精髓在于让已经得到的信息表达在先验概率之中,这也非常像我们人脑学习知识的过程。比如我们发现云很多时候大下雨情况非常多,这就是我们的先验,我们可以把自己的认识加载到先验之中,达到学习的目的。

贝叶斯公式含义:通过数据推算模型参数的概率。即:
\begin{equation}
    P({\rm{Model}} (\theta)|{\rm{Data}})=
    P({\rm{Data}}|{\rm{Model}}(\theta))P(\theta)
\end{equation}
贝叶斯统计的优势:将这个某种程度上是主观性的信息明确表达在先验概率中,而不是隐藏在没有明确指出的假设中; 让数据说话,减少主观性的先验概率。

%单变量贝叶斯参数估计
\section{贝叶斯单参数估计}
\subsection{二项分布估计}

\subsubsection{\textbf{无信息先验}}

\begin{itemize}
\item
  似然:
\end{itemize}

\begin{equation} 
p(y|\theta)=
\left(                 
  \begin{array}{ccc}   
    n\\ 
    y\\  
  \end{array}
\right)                 
\theta^y(1-\theta)^{n-y}
\end{equation}

\begin{itemize}
\item
  先验:均匀分布
\item
  后验:
\end{itemize}

\begin{equation}
  p(\theta|y)\propto\theta^y(1-\theta)^{n-y}\sim Beta(y+1,n-y+1)
\end{equation}

\begin{itemize}
\item
  预测:
\end{itemize}

\begin{equation}
  Pr(\widetilde y=1|y)=\int_0^1Pr(\widetilde y=1|\theta,y)
  =\int_0^1\theta p(\theta|y)d\theta=E(\theta|y)
\end{equation}

\subsubsection{\textbf{有信息先验}}

\begin{itemize}
\item
  先验:\(p(\theta)\propto \theta^{\alpha-1}(1-\theta)^{\beta-1}\)
\item
  似然:
\end{itemize}
\begin{equation}      
p(y|\theta)=
\left(                 
  \begin{array}{ccc}   
    n\\ 
    y\\  
  \end{array}
\right)                 
\theta^y(1-\theta)^{n-y}
\end{equation}

\begin{itemize}
\item
  后验:
\end{itemize}

\[p(\theta|y)\propto \theta^{y+\alpha-1}(1-\theta)^{n-y+\beta-1}\sim Beta(\alpha+y,\beta+n-y)\]

\begin{itemize}
\item
  后验期望:\(E(\theta|y)=\frac{\alpha+y}{\alpha+\beta+n}\)
\item
  先验期望:\(E(y)=\frac{\alpha}{\alpha+\beta}\)
\end{itemize}

当\(n\rightarrow\infty\): \(E(\theta|y)\rightarrow y/n\)

数据很大的时候可以用正态分布近似后验分布

\subsection{正态分布参数估计}

\begin{enumerate}
\def\labelenumi{\arabic{enumi}.}
\item
  \textbf{已知方差求均值}
\end{enumerate}

\begin{itemize}
\item
  似然:
\end{itemize}

\begin{equation}
  p(y|\theta)=\frac{1}{\sqrt{2\pi\sigma^2}}
  \exp\left[-\frac{1}{2\sigma^2}(y-\theta)^2\right]
  \sim N(\theta,\sigma^2)
\end{equation}


\begin{itemize}
\item
  先验:

\begin{equation}
  p(\theta)\propto\exp\left(-\frac{1}{2\tau_0^2}(\theta-\mu_0)^2\right)
  \sim N(\mu_0,\tau_0^2)
\end{equation}
\item
  后验:

\begin{equation}
  p(\theta|y)
  \propto\exp\left(-\frac{1}{2\tau_1^2}(\theta-\mu_1)^2\right)
  \sim N(\mu_1,\tau_1)
\end{equation}

  其中:

  \[\mu_1=\frac
  {\frac{1}{\tau_0^2}\mu_0+\frac{1}{\sigma^2}y}
  {\frac{1}{\tau_0^2}+\frac{1}{\sigma^2}}\]

  \[\frac{1}{\tau_1^2}=\frac{1}{\tau_0^2}+\frac{1}{\sigma^2}\]

  精度:方差的倒数
\item
  预测:

\begin{equation}
  \begin{aligned}
    p(\widetilde y|y)&=\int p(\widetilde y | \theta)p(\theta|y)d\theta\\
    &\propto
    \int\exp\left(-\frac{(\widetilde y - \theta)^2}{2\sigma^2}\right)
    \exp \left(-\frac{(\theta-\mu_1)^2}{2\tau_1^2}\right)d\theta\\
    E(\widetilde y|\theta)&=\theta\\
    D(\widetilde y|\theta)&=\sigma^2+\tau^2_1
    \end{aligned}
\end{equation}

\item
  多个相互独立数据:
\end{itemize}


\begin{equation}
  \begin{aligned}
  p(\theta \mid y) & \propto p(\theta) p(y \mid \theta) 
  \\&=p(\theta) \prod_{i=1}^{n} p\left(y_{i} \mid \theta\right) 
  \\& \propto \exp \left(-\frac{1}{2 \tau_{0}^{2}}\left(\theta-\mu_{0}\right)^{2}\right) \prod_{i=1}^{n} \exp \left(-\frac{1}{2 \sigma^{2}}\left(y_{i}-\theta\right)^{2}\right) 
  \\& \propto \exp \left(-\frac{1}{2}\left[\frac{1}{\tau_{0}^{2}}\left(\theta-\mu_{0}\right)^{2}+\frac{1}{\sigma^{2}} \sum_{i=1}^{n}\left(y_{i}-\theta\right)^{2}\right]\right)
  \end{aligned}
\end{equation}
可得:

\[p(\theta|y_1,\cdots,y_n)=p(\theta|\bar y)=N(\theta | \mu_n,\tau_n^2)\]

其中:

\[\mu_{n}=\frac{\frac{1}{\tau_{0}^{2}} \mu_{0}+\frac{n}{\sigma^{2}} \bar{y}}{\frac{1}{\tau_{0}^{2}}+\frac{n}{\sigma^{2}}}\]

\[\frac{1}{\tau_{n}^{2}}=\frac{1}{\tau_{0}^{2}}+\frac{n}{\sigma^{2}}\]

若\(n\rightarrow+\infty\),\(\tau_0\)不变,则:\(\theta|y\sim N(\bar{y},\sigma^2/n)\)

若\(\tau\rightarrow+\infty\),\(n\)
不变,则:\(\theta|y\sim N(\bar{y},\sigma^2/n)\)

\begin{enumerate}
\def\labelenumi{\arabic{enumi}.}
\item
  \textbf{已知均值求方差}
\end{enumerate}

由于有\(n\)个服从\(\sim N(\theta,\sigma^2)\)的分布,因此:

\begin{itemize}
\item
  似然:
\end{itemize}

\begin{equation}
  p(y|\sigma^2)\propto\sigma^{-n}\exp\left(-
\frac{1}{2\sigma^2}\sum_{i=1}^{n}(y_i-\theta)^2
\right)=(\sigma^2)^{-n/2}e^{-\frac{n}{2\sigma^2}v}
\end{equation}


其中:

\[v=\frac{1}{n}\sum_{i=1}^n(y_i-\theta)^2\]

\begin{itemize}
\item
  共轭先验:\(p(\sigma^2)\propto(\sigma^2)^{-\alpha+1}e^{\beta/\sigma^2} \sim \rm Inv-\chi^2(\upsilon_0,\sigma^2_0)\)
\item
  后验:
\end{itemize}

\begin{equation}
    \begin{aligned}
        p(\sigma^2|y)&\propto p(\sigma^2)p(y|\sigma^2)\\
        &\propto \left( \frac{\sigma^2_0}{\sigma^2} \right)^{v_0/2+1}
        \exp\left( -\frac{\upsilon_0\sigma^2_0}{2\sigma^2} \right)\cdot
        (\sigma^2)^{-n/2}\exp\left(-\frac{n}{2}\frac{\upsilon}{\sigma^2}\right)\\
        &\propto (\sigma^2)^{-((n+\upsilon_0)/2+1)}\exp\left(
        -\frac{1}{2\sigma^2}(\upsilon_0\sigma_0^2+n\upsilon)
        \right)\\
        &\sim \textrm{Inv}-\chi^2(\upsilon_0+n,\frac{\upsilon_0\sigma_0^2+n\upsilon}{\upsilon_0+n})
        \end{aligned}
\end{equation}

%多变量贝叶斯参数估计
\section{贝叶斯多参数模型}
\subsection{多参数模型处理}

\begin{enumerate}
\def\labelenumi{\arabic{enumi}.}
\item
  \textbf{置之不理}
\item
  \textbf{边缘化}
\begin{equation}
  p(\theta_1|y)=\int p(\theta_1,\theta_2|y)d\theta_2
\end{equation}
  
  用贝叶斯公式展开:

  \[p(\theta_1,\theta_2|y)\propto p(y|\theta_1,\theta_2)p(\theta_1,\theta_2)\]

  将\(\theta_2\)边缘化积分,得到\(\theta_1\)的后验分布。
\item
  \textbf{平均化}

  \[p(\theta_1|y)=\int p(\theta_1|\theta_2,y)p(\theta_2|y)d\theta_2\]
\end{enumerate}

\subsection{无信息先验的正态分布}

\begin{itemize}
\item
  先验:

\begin{equation}
  p(\mu,\ln\sigma^2)\sim U(\mu,\ln \sigma^2)
\end{equation}

  或者先验写为:

  \[p(\mu,\sigma^2)\sim\frac{1}{\sigma^2}\]
\item
  似然 ( \textbf{有\(n\)次观测 }) :

  \[p(y|\mu,\sigma^2)=\sigma^{-n}\exp \left(
  -\frac{1}{2\sigma^2}\sum_{i=1}^{n}(y_i-\mu)^2
  \right)\]
\item
  联合后验:

  \[p(\mu,\sigma^2|y)\propto \sigma^{-n-2}\exp\left(
  -\frac{1}{2\sigma^2}[(n-1)s^2+n(\bar{y}-\mu)^2]
  \right)\]

  其中:

  \[s^2=\frac{1}{n-1}\sum_{i=1}^{n}(y_i-\bar y)^2\]
\item
  边缘后验\(p(\sigma^2|y)\):

\begin{equation}
  \begin{aligned}
  p(\sigma^2|y)
  &\propto \int p(\mu,\sigma^2|y)d\mu\\
  &\propto(\sigma^2)^{-\frac{n+1}{2}}\exp\left(
  -\frac{(n-1)s^2}{2\sigma^2}
  \right)\\
  &\sim Inv-\chi^2(n-1,s^2)
  \end{aligned}
\end{equation}

\item
  边缘后验\(p(\mu|y)\):

  \begin{align*}
  p(\mu|y)
  &=\int_0^{\infty}p(\mu,\sigma^2|y)d\sigma^2\\
  &\propto\left[
  1+\frac{n(\mu-\bar y)^2}{(n-1)s^2}^{-n/2}
  \right] \\
  &\sim t_{n-1}(\bar y,s^2/n)
  \end{align*}
\item
  预测后验分布

  \[p(\widetilde y|y)=\iint  p(\widetilde y |\mu,\sigma^2,y)p(\mu,\sigma^2|y)
  d\mu d\sigma^2\]
\end{itemize}

\subsection{共轭先验分布}

\begin{itemize}
\item
  先验分布:

  \(\mu|\sigma^2 \sim N(\mu_0,\sigma^2/\kappa_0)\)

  \(\sigma^2 \sim Inv-\chi^2(\nu_0,\sigma^2_0)\)
\item
  联合先验分布

  \begin{align*}
  p(\mu,\sigma^2)
  &=p(\mu|\sigma^2)p(\sigma^2)\\
  &\propto N-Inv-\chi^2(\mu_0,\sigma^2_0/\kappa_0\ ;\ \nu_0,\sigma^2_0 )
  \end{align*}
\item
  似然分布

  \[p(y|\mu,\sigma^2)=\sigma^{-n}\exp \left(
  -\frac{1}{2\sigma^2}\sum_{i=1}^{n}(y_i-\mu)^2
  \right)\]
\item
  联合后验分布
\begin{equation}
  \begin{aligned}
    p\left(\mu, \sigma^{2} \mid y\right) 
    & \propto \sigma^{-1}\left(\sigma^{2}\right)^{\left(\nu_{0} / 2+1\right)} e^{-\frac{1}{2 \sigma^{2}}\left[\nu_{0} \sigma_{0}^{2}+\kappa_{0}\left(\mu-\mu_{0}\right)^{2}\right]}
    \left(\sigma^{2}\right)^{-n / 2} e^{-\frac{1}{2 \sigma^{2}}\left[(n-1) s^{2}+n(\bar{y}-\mu)^{2}\right]} \\
    & \propto \mathrm{N}-\operatorname{Inv}-\chi^{2}\left(\mu_{n}, \sigma_{n}^{2} / \kappa_{n} ; \nu_{n}, \sigma_{n}^{2}\right)
  \end{aligned}
\end{equation}


\end{itemize}

其中参数为:
\begin{equation}
  \begin{array}{c}
    \mu_{n}   = \frac{\kappa_{0}}{\kappa_{0}+n} \mu_{0}+\frac{n}{\kappa_{0}+n} \bar{y}\\
    \kappa_{n}= \kappa_{0}+n \\
    \nu_{n}   = \nu_{0}+n \\
    \nu_{n} \sigma_{n}^{2}=\nu_{0} \sigma_{0}^{2}+(n-1) s^{2}+\frac{\kappa_{0} n}{\kappa_{0}+n}\left(\bar{y}-\mu_{0}\right)^{2}
    \end{array}
\end{equation}


\begin{itemize}
\item
  条件后验分布\(p(\mu|\sigma^2,y)\)
\end{itemize}

\[(\mu \mid \sigma^{2}, y) \sim \mathrm{N}\left(\mu_{n} \frac{\sigma^{2}}{\kappa_{n}}\right)\]

\begin{itemize}
\item
  方差边缘后验分布\(p(\sigma^2|y)\)
\end{itemize}

\[(\sigma^{2} \mid y) \sim \operatorname{Inv}-\chi^{2}\left(\nu_{n}, \sigma_{n}^{2}\right)\]

\begin{itemize}
\item
  均值边缘后验分布\(p(\mu|y)\)
\end{itemize}

\begin{equation}
  \begin{aligned}
    p(\mu \mid y) 
    & \propto\left[1+\frac{\kappa_{n}\left(\mu-\mu_{n}\right)^{2}}{\nu_{n} \sigma_{n}^{2}}\right]^{-\left(\nu_{n}+1\right) / 2} \\
    & = t_{\nu_{n}}\left(\mu \mid m u_{n}, \sigma_{n}^{2} / \kappa_{n}\right)
  \end{aligned}
\end{equation}

%分层贝叶斯模型
\section{层次化贝叶斯模型}

\subsection{参数化先验分布}
\begin{enumerate}
\def\labelenumi{\arabic{enumi}.}
\item
  先验分布\cite{gelman_bayesian_2014}

  先验分布是由某个未知参数的分布\(\phi\)给出:

  \[p(\theta|\phi)=\prod_{j=1}^Jp(\theta_j|\phi)\ \ \ \ \  \theta=(\theta_1,\theta_2,\cdots)\]

  边缘化:

  \[p(\theta)=\int\left( \prod_{j=1}^Jp(\theta_j|\phi) \right)p(\phi)d\phi\]
\item
  联合先验分布 \(p(\phi,\theta)=p(\theta|\phi)p(\phi)\)
\item
  超先验分布 : \(p(\phi)\)
\item
  后验分布

  \begin{itemize}
  \item
    联合后验 : \(p(\phi,\theta|y)\propto p(y|\theta)p(\theta|\phi)p(\phi)\)
  \item
    条件后验:\(p(\theta|\phi,y)\)
  \item
    边缘后验:\(p(\phi|y)\)

    \[p(\phi|y)=\frac{p(\theta,\phi|y)}{p(\theta|\phi,y)}\]
  \end{itemize}

\item 层次化贝叶斯完整表述
\begin{equation}
  \begin{aligned}
    p(\phi,\theta|y)
    &\propto p(y|\phi,\theta)p(\phi,\theta)\\
    &=p(y|\theta)p(\phi,\theta)\\
    &=p(y|\theta)p(\theta|\phi)p(\phi)
    \end{aligned}
\end{equation}
由于likelihood中$p(y|\phi,\theta)$只取决于$\theta$,因此超参数$\phi$只通过参数$\theta$影响y。
\item 层次化贝叶斯计算步骤

\begin{itemize}
\item
  写出联合后验分布\(p(\theta,\phi|y)\):即超先验分布,总体分布和似然分布的乘积
\item
  确定条件后验分布\(p(\theta|\phi,y)\):

  \[p(\theta|\phi,y)=\prod_{j=1}^J p(\theta_j|\phi,y)\]
\item
  边缘化给出\(\phi\)的贝叶斯估计
\end{itemize}
\end{enumerate}

\subsection{二项分布的分层贝叶斯模型}

\begin{enumerate}
\def\labelenumi{\arabic{enumi}.}
\item \(y_i\)先验(组内模型) \(y_j \sim Bin(n_j,\theta_j)\)
\item \(\theta_j\)先验(组间模型):\(\theta_j\sim Beta(\alpha,\beta)\)
\item 联合先验:
      \(p(\alpha,\beta,\theta)=p(\alpha,\beta)p(\theta|\alpha,\beta)\)
\item 似然:
      \(p(y|\theta,\alpha,\beta)\)
\item 联合后验:
      \(p(\theta,\alpha,\beta|y)\)

\begin{equation}
  \begin{aligned}
    p(\theta,\alpha,\beta|y)
    &\propto p(\alpha,\beta)p(\theta|\alpha,\beta)p(y|\theta,\alpha,\beta)\\
    &=p(\alpha,\beta)\prod_{j=1}^{J}
    \frac{\Gamma(\alpha+\beta)}{\Gamma(\alpha)\Gamma(\beta)}
    \theta_j^{\alpha-1}(1-\theta_j)^{\beta-1}
    \prod_{j=1}^{J}
    \theta_j^{y_j}(1-\theta_j)^{n_j-y_j}
    \end{aligned}
\end{equation}

\item
  条件后验: \(p(\theta|\alpha,\beta,y)\) : 单参数模型给定的后验

\begin{equation}
  \begin{aligned}
    p(\theta \mid \alpha, \beta, y)
    &=\prod_{j=1}^{J} \frac{\Gamma\left(\alpha+\beta+n_{j}\right)}{\Gamma\left(\alpha+y_{j}\right) \Gamma\left(\beta+n_{j}-y_{j}\right)} 
    \theta_{j}^{\alpha+y_{j}-1}\left(1-\theta_{j}\right)^{\beta+n_{j}-y_{j}-1}\\
    &\sim \prod _{j=1}^JBeta(\alpha+y_j,\beta+n_j-y_j)
    \end{aligned}
\end{equation}

\item
  边缘后验:\(p(\alpha,\beta|y)\)
\begin{equation}
  \begin{aligned}
    p(\alpha, \beta \mid y)
    =& \frac{p(\theta, \alpha, \beta \mid y)}{p(\theta \mid \alpha, \beta, y)} \\
    &\propto \frac{p(\alpha, \beta) \prod_{j=1}^{J} 
    \frac{\Gamma(\alpha+\beta)}{\Gamma(\alpha) \Gamma(\beta)} 
    \theta_{j}^{\alpha-1}\left(1-\theta_{j}\right)^{\beta-1} \prod_{j=1}^{J} \theta_{j}^{y_{j}}\left(1-\theta_{j}\right)^{n_{j}-y_{j}}}{\prod_{j=1}^{J} 
    \frac{\Gamma\left(\alpha+\beta+n_{j}\right)}{\Gamma\left(\alpha+y_{j}\right) \Gamma\left(\beta+n_{j}-y_{j}\right)} 
    \theta_{j}^{\alpha+y_{j}-1}\left(1-\theta_{j}\right)^{\beta+n_{j}-y_{j}-1}} \\
    =& p(\alpha, \beta) \prod_{j=1}^{J} \frac{\Gamma(\alpha+\beta) \Gamma\left(\alpha+y_{j}\right) 
    \Gamma\left(\beta+n_{j}-y_{j}\right)}{\Gamma(\alpha) \Gamma(\beta) \Gamma\left(\alpha+\beta+n_{j}\right)}
    \end{aligned}
\end{equation}

\end{enumerate}


\subsection{正态分布的分层贝叶斯模型}

\begin{enumerate}
\def\labelenumi{\arabic{enumi}.}
\item
  \textbf{数据结构}


假设\(J\)个独立试验,每个实验都由\(\theta_j\)给出其参数估计,估计\(n_j\)个\(i.i.d\)正态分布的数据点\(y_{ij}\),每个点方差为\(\sigma^2\):

\[y_{i j} | \theta_{j} \sim N\left(\theta_{j}, \sigma^{2}\right) \quad i=1, \ldots, n_{j} \quad j=1, \ldots, J\]

样本均值(充分统计量):

\[\bar{y}_{\cdot j}=\frac{1}{n_j}\sum_{i=1}^{n_j}y_{ij}\]

样本均值的分布

\[\bar y_{\cdot j} \sim N(\theta_j,\sigma^2_j)\]

样本方差:

\[\sigma^2_j=\frac{\sigma^2}{n_j}\]

样本的均值是从\(\theta\)中估计,样本方差是从\(\sigma\)中估计。

相当于 \(\theta\) 的似然分布:

\[\bar y_{\cdot j}|\theta \sim N(\theta_j,\sigma^2_j)\]

由于 \(\sigma\) 是已知的,下面所有的分布都是在 \(\sigma\)
已知情况下成立。


\item
  \textbf{层次化模型:无信息先验}


\(\theta\)是从参数\((\mu,\tau)\)中抽取:

\[p\left(\theta_{1}, \ldots, \theta_{J} \mid \mu, \tau^{2}\right)=\prod_{j=1}^{J} p\left(\theta_{j} \mid \mu, \tau^{2}\right)\]

边缘化:

\[p\left(\theta_{1}, \ldots, \theta_{J}\right)=
\iint \prod_{j=1}^{J}
\left[p\left(\theta_{j} \mid \mu, \tau^{2}\right)\right] 
p\left(\mu, \tau^{2}\right)
d \mu d \tau\]

\begin{itemize}
\item
  \textbf{先验和似然}
\end{itemize}

组内抽样:\(\bar y_{\cdot j} \sim N(\theta_j,\sigma^2_j)\)

\(\theta\)的先验(组内模型):\(\theta|\mu,\tau \sim N(\mu,\tau^2)\)

\(\mu\)的先验: \(p(\mu,\tau)=p(\mu|\tau)p(\tau)\propto p(\tau)\)

\(\theta_j\)似然分布:\(p(y|\theta)\sim(\bar y_{\cdot j}|\theta) \sim N(\theta_j,\sigma^2_j)\)

\begin{itemize}
\item
  \textbf{联合后验:}\(p(\theta,\phi|y)\)
\end{itemize}

\begin{equation}
  \begin{aligned}
    p(\theta,\mu,\tau|y)
    &\propto p(\mu,\tau)p(\theta|\mu,\tau)p(y|\theta)\\
    &\propto p(\mu,\tau)\prod_{j=1}^{J}p(\theta_j|\mu,\tau^2)
    \prod_{j=1}^{J}p(\bar y _{\cdot j}|\theta_j,\sigma^2_j)
    \end{aligned}
\end{equation}


其中:\(\bar y _{\cdot j}\sim N(\theta_j,\sigma_j^2)\)

可以忽略只依赖 \(y\) 和 \(\sigma_j\) 的参数,因为其已知。

\begin{itemize}
\item
  \textbf{\(\theta\)条件后验:}\(p(\theta_{j} \mid \mu, \tau, y_{\cdot, j})\)
\end{itemize}

\[\theta_{j} \mid \mu, \tau, y_{\cdot, j} \sim N\left(\hat{\theta}_{j}, V_{j}\right)\]

其中:

\[\hat{\theta}_{j}=
\frac
{\frac{1}{\sigma_{j}^{2}} \bar{y}_{\cdot j}+
\frac{1}{\tau^{2}} \mu}
{\frac{1}{\sigma_{j}^{2}}+\frac{1}{\tau^{2}}}\]

\[V_{j}=\frac{1}{\frac{1}{\sigma_{j}^{2}}+\frac{1}{\tau^{2}}}\]

\begin{itemize}
\item
  \textbf{超参数边缘后验:}\(p(\mu, \tau \mid y)\)
\end{itemize}

\[p(\mu, \tau \mid y) \propto p(\mu, \tau) p(y \mid \mu, \tau)\]

对于正态分布:

\[\bar{y}_{\cdot j}\sim N(\mu,\tau^2+\sigma^2)\]

因此:
\begin{equation}
  p(\mu, \tau \mid y) \propto p(\mu, \tau) 
  \prod_{j=1}^{J} p\left(\bar{y} _{\cdot j} \mid \mu, \tau^{2}+\sigma_{j}^{2}\right)
\end{equation}

\begin{equation}
  \bar{y}_{\cdot j} \mid \mu, (\tau^{2}+\sigma_{j}^{2}) 
  \sim
  N \left( \mu, \tau^{2} + \sigma_{j}^{2} \right)
\end{equation}

\begin{itemize}
\item
  \textbf{给定\(\tau\)下\(\mu\)的边缘后验分布}:
  \(p(\mu \mid \tau, y)\)从单参数模型中得出的结论
\end{itemize}

\begin{equation}
  \begin{array}{c}
    \mu \mid \tau, y \sim N\left(\hat{\mu}, V_{\mu}\right) \\
    \end{array}
\end{equation}


\[\hat{\mu}=\frac{\sum_{j=1}^{J} \frac{1}{\sigma_{j}^{2}+\tau^{2}} \bar{y}_{\cdot j}}{\sum_{j=1}^{J} \frac{1}{\sigma_{j}^{2}+\tau^{2}}}\]

\[\\V_{\mu}^{-1}=\sum_{j=1}^{J} \frac{1}{\sigma_{j}^{2}+\tau^{2}}\]

\begin{itemize}
\item
  \(\tau\)的后验:\(p(\tau \mid y) \)
\end{itemize}

\begin{equation}
  \begin{aligned}
    p(\tau \mid y) 
    &=\frac{p(\mu, \tau \mid y)}{p(\mu \mid \tau, y)} \\
    &\propto 
    \frac{p(\tau) \prod_{j=1}^{J} 
    N\left(\bar{y}_{ \cdot j }\mid \mu, \sigma_{j}^{2}+\tau^{2}\right)}
    {N\left(\mu \mid \hat{\mu}, V_{\mu}\right)} \\
    & \propto p(\tau) V_{\mu}^{1 / 2} 
    \prod_{j=1}^{J}\left(\sigma_{j}^{2}+\tau^{2}\right)^{-1 / 2} 
    \exp \left(
    -\frac{\left(\bar{y}_{\cdot j}-\hat{\mu}\right)^{2}}{2\left(\sigma_{j}^{2}+\tau^{2}\right)}
    \right)
    \end{aligned}
\end{equation}

从后往前采样,即先算出\(\tau\)的后验,然后依次采样算出\(\mu\)的后验,超参数联合后验,\(\theta\)的条件后验,联合后验等。


\end{enumerate}

\section{贝叶斯回归}
\subsection{线性贝叶斯回归}
对于有n组观测数据,一般记为:
\begin{equation}
    y=(y_1,y_2,\cdots,y_n)^T
\end{equation}
解释变量记为:
\begin{equation}
    X=\begin{bmatrix}  
        1 & x_{11} & x_{12} & \cdots & x_{1k} \\
        1 & x_{21} & x_{22} & \cdots & x_{2k} \\
        1 & \vdots & \vdots & \ddots & \vdots \\
        1 & x_{n1} & x_{n2} & \cdots & x_{nk}
      \end{bmatrix}
\end{equation}
我们看到一共有n行,也就是一共采样到n组数据;有k列,也就是说一组解释数据有(k-1)个变量。一般把第一列设置为1,因为我们在进行线性回归时有一个常数。回归系数$\beta = (\beta_1,\beta_2,\cdots,\beta_k)^T$。这样回归方程可以写为:
\begin{equation}
    E(y_i|\beta,X)=\beta_1 x_{i1}+\cdots+\beta_kx_{ik} \qquad i=1,\cdots,n
\end{equation}
用矩阵表示为:
\begin{equation}
    E(y|\beta,X)=X\beta
\end{equation}
对于常规的线性回归情形,一般的我们认为给定的回归残差为一个正态分布,其回归方差为:
\begin{equation*}
    D(y_i|\theta,X)=\sigma^2
\end{equation*}
对于所有的i组数据都相同。因此我们的回归参数$\theta=(\beta,\sigma^2)$我们记:
\begin{equation}
    p(y|\beta,\sigma^2,X)\sim N(X\beta,\sigma^2I)
\end{equation}
其中I是一个$n\times n$的单位矩阵。对于正态回归模型,我们的先验可以写为:
\begin{equation}
    p(\beta,\sigma^2|X)\sim \sigma^{-2}
\end{equation}

\subsection{更一般的贝叶斯回归}
\subsection{有先验信息的贝叶斯回归}
\section{贝叶斯模型选择}
%在模型选择上贝叶斯因子有非常重要的作用,贝叶斯因子说明数据更加偏向哪个假设。这部分参考了MIT的Jeremy Orloff and Jonathan Bloom\thanks{\href{MIT:Jeremy Orloff and Jonathan Bloom}{https://ocw.mit.edu/courses/mathematics/18-05-introduction-to-probability-and-statistics-spring-2014/readings/MIT18_05S14_Reading12b.pdf}}的讲义
\begin{definition}{贝叶斯因子\\}
    在观测数据d中选择两个模型假设$H_1$和$H_2$,其参数为$\theta_1$和$\theta_2$,贝叶斯因子为:
    \begin{equation}
        \mathcal{B}_{12}=\frac{p(d|H_1(\theta_1))}{p(d|H_2(\theta_2))} \label{bayes_factor}
    \end{equation}
    下面把贝叶斯因子(Bayes factor)缩写为BF。
\end{definition}
当$\mathcal{B}_{12}>1$时支持假设$H_1$,反之支持假设$H_2$。当然BF也有相应的判据,下面会慢慢介绍。
\begin{definition}{odds ratio\\}
    对于事件E和事件E的补集$E^c$,其概率比值:
    \begin{equation}
        \mathcal{O}(E)=\frac{P(E)}{P(E^c)}
    \end{equation}
    称为\textrm{odds ratio}(下面缩写为OR)。从数据分析的角度上说,有两个模型假设$H_1$和$H_2$,OR为:
    \begin{equation}
        \mathcal{O}_{12}=\frac{p(H_1|d)}{p(H_2|d)} \label{ORs}
    \end{equation}
\end{definition}
OR和BF的关系是什么呢?结合BF公式\ref{bayes_factor}我们把OR展开:
\begin{equation}
    \begin{aligned}
        \mathcal{O}_{12} &= \frac{p(H_1|d)}{p(H_2|d)}\\
        &= \frac{p(d|H_1)p(H_1)}{p(d|H_2)p(H_2)}\\
        &= \mathrm{BF}\cdot\frac{p(H_1)}{p(H_2)}\\
        &= \mathrm{BF}\cdot \mathrm{prior\ odds}
    \end{aligned}
\end{equation}
因此Odds ratio是Bayes factor乘先验比。

对于模型选择,有一个非常著名的定理:\textbf{奥卡姆剃刀}。这个定理的概括来说就是:\textbf{如无必要,勿增实体}。
\section{费舍尔信息矩阵Fisher Information}

\begin{equation}
    \Gamma_{ij}(\theta):=E_{\theta}\left\{ \frac{\partial \ln p(x;\boldsymbol{\theta})}{\partial \theta_i} \frac{\partial \ln p(x;\boldsymbol{\theta})}{\partial \theta_j} \right\}
\end{equation}