\section{Chapter 5层次化模型}

\subsubsection{一、参数化先验分布}
\begin{enumerate}
\def\labelenumi{\arabic{enumi}.}
\item
  先验分布

  先验分布是由某个未知参数的分布\(\phi\)给出:

  \[p(\theta|\phi)=\prod_{j=1}^Jp(\theta_j|\phi)\ \ \ \ \  \theta=(\theta_1,\theta_2,\cdots)\]

  边缘化:

  \[p(\theta)=\int\left( \prod_{j=1}^Jp(\theta_j|\phi) \right)p(\phi)d\phi\]
\item
  联合先验分布 \(p(\phi,\theta)=p(\theta|\phi)p(\phi)\)
\item
  超先验分布 : \(p(\phi)\)
\item
  后验分布

  \begin{itemize}
  \item
    联合后验 : \(p(\phi,\theta|y)=p(y|\theta)p(\theta|\phi)p(\phi)\)
  \item
    条件后验:\(p(\theta|\phi,y)\)
  \item
    边缘后验:\(p(\phi|y)\)

    \[p(\phi|y)=\frac{p(\theta,\phi|y)}{p(\theta|\phi,y)}\]
  \end{itemize}
\item
  层次化贝叶斯完整表述
\end{enumerate}

\begin{equation}
  \begin{aligned}
    p(\phi,\theta|y)
    &\propto p(y|\phi,\theta)p(\phi,\theta)\\
    &=p(y|\theta)p(\phi,\theta)\\
    &=p(y|\theta)p(\theta|\phi)p(\phi)
    \end{aligned}
\end{equation}


\begin{enumerate}
\def\labelenumi{\arabic{enumi}.}
\item
  层次化贝叶斯计算步骤
\end{enumerate}

\begin{itemize}
\item
  写出联合后验分布\(p(\theta,\phi|y)\):即超先验分布,总体分布和似然分布的乘积
\item
  确定条件后验分布\(p(\theta|\phi,y)\):

  \[p(\theta|\phi,y)=\prod_{j=1}^J p(\theta_j|\phi,y)\]
\item
  边缘化给出\(\phi\)的贝叶斯估计
\end{itemize}

\hypertarget{ux4e8cux4e8cux9879ux5206ux5e03ux7684ux5206ux5c42ux8d1dux53f6ux65afux6a21ux578b}{%
\subsubsection{二、二项分布的分层贝叶斯模型}\label{ux4e8cux4e8cux9879ux5206ux5e03ux7684ux5206ux5c42ux8d1dux53f6ux65afux6a21ux578b}}

\begin{enumerate}
\def\labelenumi{\arabic{enumi}.}
\item
  \(y_i\)先验(组内模型) \(y_j \sim Bin(n_j,\theta_j)\)
\item
  \(\theta_j\)先验(组间模型):\(\theta_j\sim Beta(\alpha,\beta)\)
\item
  联合先验:\(p(\alpha,\beta,\theta)=p(\alpha,\beta)p(\theta|\alpha,\beta)\)
\item
  似然:\(p(y|\theta,\alpha,\beta)\)
\item
  联合后验:\(p(\theta,\alpha,\beta|y)\)

\begin{equation}
  \begin{aligned}
    p(\theta,\alpha,\beta|y)
    &\propto p(\alpha,\beta)p(\theta|\alpha,\beta)p(y|\theta,\alpha,\beta)\\
    &=p(\alpha,\beta)\prod_{j=1}^{J}
    \frac{\Gamma(\alpha+\beta)}{\Gamma(\alpha)\Gamma(\beta)}
    \theta_j^{\alpha-1}(1-\theta_j)^{\beta-1}
    \prod_{j=1}^{J}
    \theta_j^{y_j}(1-\theta_j)^{n_j-y_j}
    \end{aligned}
\end{equation}

\item
  条件后验: \(p(\theta|\alpha,\beta,y)\) : 单参数模型给定的后验

\begin{equation}
  \begin{aligned}
    p(\theta \mid \alpha, \beta, y)
    &=\prod_{j=1}^{J} \frac{\Gamma\left(\alpha+\beta+n_{j}\right)}{\Gamma\left(\alpha+y_{j}\right) \Gamma\left(\beta+n_{j}-y_{j}\right)} 
    \theta_{j}^{\alpha+y_{j}-1}\left(1-\theta_{j}\right)^{\beta+n_{j}-y_{j}-1}\\
    &\sim \prod _{j=1}^JBeta(\alpha+y_j,\beta+n_j-y_j)
    \end{aligned}
\end{equation}

\item
  边缘后验:\(p(\alpha,\beta|y)\)
\begin{equation}
  \begin{aligned}
    p(\alpha, \beta \mid y)
    =& \frac{p(\theta, \alpha, \beta \mid y)}{p(\theta \mid \alpha, \beta, y)} \\
    &\propto \frac{p(\alpha, \beta) \prod_{j=1}^{J} 
    \frac{\Gamma(\alpha+\beta)}{\Gamma(\alpha) \Gamma(\beta)} 
    \theta_{j}^{\alpha-1}\left(1-\theta_{j}\right)^{\beta-1} \prod_{j=1}^{J} \theta_{j}^{y_{j}}\left(1-\theta_{j}\right)^{n_{j}-y_{j}}}{\prod_{j=1}^{J} 
    \frac{\Gamma\left(\alpha+\beta+n_{j}\right)}{\Gamma\left(\alpha+y_{j}\right) \Gamma\left(\beta+n_{j}-y_{j}\right)} 
    \theta_{j}^{\alpha+y_{j}-1}\left(1-\theta_{j}\right)^{\beta+n_{j}-y_{j}-1}} \\
    =& p(\alpha, \beta) \prod_{j=1}^{J} \frac{\Gamma(\alpha+\beta) \Gamma\left(\alpha+y_{j}\right) 
    \Gamma\left(\beta+n_{j}-y_{j}\right)}{\Gamma(\alpha) \Gamma(\beta) \Gamma\left(\alpha+\beta+n_{j}\right)}
    \end{aligned}
\end{equation}

\end{enumerate}

\hypertarget{ux4e09ux6b63ux6001ux5206ux5e03ux7684ux5206ux5c42ux8d1dux53f6ux65afux6a21ux578b}{%
\subsubsection{三、正态分布的分层贝叶斯模型}\label{ux4e09ux6b63ux6001ux5206ux5e03ux7684ux5206ux5c42ux8d1dux53f6ux65afux6a21ux578b}}

\begin{enumerate}
\def\labelenumi{\arabic{enumi}.}
\item
  \textbf{数据结构}
\end{enumerate}

假设\(J\)个独立试验,每个实验都由\(\theta_j\)给出其参数估计,估计\(n_j\)个\(i.i.d\)正态分布的数据点\(y_{ij}\),每个点方差为\(\sigma^2\):

\[y_{i j} | \theta_{j} \sim N\left(\theta_{j}, \sigma^{2}\right) \quad i=1, \ldots, n_{j} \quad j=1, \ldots, J\]

样本均值(充分统计量):

\[\bar{y}_{\cdot j}=\frac{1}{n_j}\sum_{i=1}^{n_j}y_{ij}\]

样本均值的分布

\[\bar y_{\cdot j} \sim N(\theta_j,\sigma^2_j)\]

样本方差:

\[\sigma^2_j=\frac{\sigma^2}{n_j}\]

样本的均值是从\(\theta\)中估计,样本方差是从\(\sigma\)中估计。

相当于 \(\theta\) 的似然分布:

\[\bar y_{\cdot j}|\theta \sim N(\theta_j,\sigma^2_j)\]

由于 \(\sigma\) 是已知的,下面所有的分布都是在 \(\sigma\)
已知情况下成立。

\begin{enumerate}
\def\labelenumi{\arabic{enumi}.}
\item
  \textbf{层次化模型:无信息先验}
\end{enumerate}

\(\theta\)是从参数\((\mu,\tau)\)中抽取:

\[p\left(\theta_{1}, \ldots, \theta_{J} \mid \mu, \tau^{2}\right)=\prod_{j=1}^{J} p\left(\theta_{j} \mid \mu, \tau^{2}\right)\]

边缘化:

\[p\left(\theta_{1}, \ldots, \theta_{J}\right)=
\iint \prod_{j=1}^{J}
\left[p\left(\theta_{j} \mid \mu, \tau^{2}\right)\right] 
p\left(\mu, \tau^{2}\right)
d \mu d \tau\]

\begin{itemize}
\item
  \textbf{先验和似然}
\end{itemize}

组内抽样:\(\bar y_{\cdot j} \sim N(\theta_j,\sigma^2_j)\)

\(\theta\)的先验(组内模型):\(\theta|\mu,\tau \sim N(\mu,\tau^2)\)

\(\mu\)的先验: \(p(\mu,\tau)=p(\mu|\tau)p(\tau)\propto p(\tau)\)

\(\theta_j\)似然分布:\(p(y|\theta)\sim(\bar y_{\cdot j}|\theta) \sim N(\theta_j,\sigma^2_j)\)

\begin{itemize}
\item
  \textbf{联合后验:}\(p(\theta,\phi|y)\)
\end{itemize}

\begin{equation}
  \begin{aligned}
    p(\theta,\mu,\tau|y)
    &\propto p(\mu,\tau)p(\theta|\mu,\tau)p(y|\theta)\\
    &\propto p(\mu,\tau)\prod_{j=1}^{J}p(\theta_j|\mu,\tau^2)
    \prod_{j=1}^{J}p(\bar y _{\cdot j}|\theta_j,\sigma^2_j)
    \end{aligned}
\end{equation}


其中:\(\bar y _{\cdot j}\sim N(\theta_j,\sigma_j^2)\)

可以忽略只依赖 \(y\) 和 \(\sigma_j\) 的参数,因为其已知。

\begin{itemize}
\item
  \textbf{\(\theta\)条件后验:}\(p(\theta_{j} \mid \mu, \tau, y_{\cdot, j})\)
\end{itemize}

\[\theta_{j} \mid \mu, \tau, y_{\cdot, j} \sim N\left(\hat{\theta}_{j}, V_{j}\right)\]

其中:

\[\hat{\theta}_{j}=
\frac
{\frac{1}{\sigma_{j}^{2}} \bar{y}_{\cdot j}+
\frac{1}{\tau^{2}} \mu}
{\frac{1}{\sigma_{j}^{2}}+\frac{1}{\tau^{2}}}\]

\[V_{j}=\frac{1}{\frac{1}{\sigma_{j}^{2}}+\frac{1}{\tau^{2}}}\]

\begin{itemize}
\item
  \textbf{超参数边缘后验:}\(p(\mu, \tau \mid y)\)
\end{itemize}

\[p(\mu, \tau \mid y) \propto p(\mu, \tau) p(y \mid \mu, \tau)\]

对于正态分布:

\[\bar{y}_{\cdot j}\sim N(\mu,\tau^2+\sigma^2)\]

因此:

\[p(\mu, \tau \mid y) \propto p(\mu, \tau) \prod_{j=1}^{J} p\left(\bar{y} _{\cdot j} \mid \mu, \tau^{2}+\sigma_{j}^{2}\right)\]

\[\bar{y}_{\cdot j} \mid \mu, (\tau^{2}+\sigma_{j}^{2}) 
\sim
N \left( \mu, \tau^{2} + \sigma_{j}^{2} \right)\]

\begin{itemize}
\item
  \textbf{给定\(\tau\)下\(\mu\)的边缘后验分布}:
  \(p(\mu \mid \tau, y)\)从单参数模型中得出的结论
\end{itemize}

\begin{equation}
  \begin{array}{c}
    \mu \mid \tau, y \sim N\left(\hat{\mu}, V_{\mu}\right) \\
    \end{array}
\end{equation}


\[\hat{\mu}=\frac{\sum_{j=1}^{J} \frac{1}{\sigma_{j}^{2}+\tau^{2}} \bar{y}_{\cdot j}}{\sum_{j=1}^{J} \frac{1}{\sigma_{j}^{2}+\tau^{2}}}\]

\[\\V_{\mu}^{-1}=\sum_{j=1}^{J} \frac{1}{\sigma_{j}^{2}+\tau^{2}}\]

\begin{itemize}
\item
  \(\tau\)的后验:\(p(\tau \mid y) \)
\end{itemize}

\begin{equation}
  \begin{aligned}
    p(\tau \mid y) 
    &=\frac{p(\mu, \tau \mid y)}{p(\mu \mid \tau, y)} \\
    &\propto 
    \frac{p(\tau) \prod_{j=1}^{J} 
    N\left(\bar{y}_{ \cdot j }\mid \mu, \sigma_{j}^{2}+\tau^{2}\right)}
    {N\left(\mu \mid \hat{\mu}, V_{\mu}\right)} \\
    & \propto p(\tau) V_{\mu}^{1 / 2} 
    \prod_{j=1}^{J}\left(\sigma_{j}^{2}+\tau^{2}\right)^{-1 / 2} 
    \exp \left(
    -\frac{\left(\bar{y}_{\cdot j}-\hat{\mu}\right)^{2}}{2\left(\sigma_{j}^{2}+\tau^{2}\right)}
    \right)
    \end{aligned}
\end{equation}


\begin{itemize}
\item
  从后往前采样,即先算出\(\tau\)的后验,然后依次采样算出\(\mu\)的后验,超参数联合后验,\(\theta\)的条件后验,联合后验等。
\end{itemize}