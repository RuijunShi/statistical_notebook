\section{马尔科夫链 Markov Chain}

\subsection{马尔科夫链原理}

{一、马尔科夫链的定义}

首先我们来看一个天气预报的例子。假设我们天气只有2种状态,一种是晴天,一种是雨天,那我我们从以前的观测发现今天是晴天,明天是晴天或者雨天的概率为:\([0.9,0.1]^T\),而今天雨天,明天晴天或者阴天的概率为:\([0.5,0.5]^T\),也就是说我们不考虑其他影响,只考虑今天对明天天气的影响。那怎么表示明天的下雨的概率呢?简单来,我们可以做一个``加权平均'',明天是晴天的概率也就是等于:

\[P_{晴}=P_{晴-晴}+P_{雨-晴}=今天晴天的概率*明天晴天的概率+今天雨天的概率*明天晴天的概率\]

同理,阴天的概率也可以相似地形式算出来。这个形式是不是很像我们的矩阵乘法?首先我们写出一个\textbf{概率转移矩阵\(P\)}(下文会讲这个矩阵的作用):

\begin{equation}
  P = 
\begin{bmatrix} 
0.9 &0.5\\0.1 & 0.5
\end{bmatrix} 
\end{equation}


\begin{figure}
\centering
%\includegraphics{C:/Users/astro_gh/Desktop/QQ图片20201130235426.png}
\caption{QQ图片20201130235426}
\end{figure}

在时刻\(t_0\)我们假设今天的天气状态是晴天,也就是说今天晴天的概率为1,雨天的概率为0:

\begin{equation}
  \pi (0)=
\begin{bmatrix} 
1\\0
\end{bmatrix} 
\end{equation}


那么我们要怎么算出明天天气是晴天或者雨天的概率呢?我们可以:
\begin{equation}
  \pi(1)  = P\pi(0) =
    \begin{bmatrix} 
    0.9 &0.5\\0.1 & 0.5
    \end{bmatrix} 
    \begin{bmatrix} 
    1\\0
    \end{bmatrix} 
    =
    \begin{bmatrix} 
    0.9\\0.1
    \end{bmatrix}
\end{equation}


这意味着今天晴天,明天晴天的概率为0.9,下雨的概率为0.1。上面的例子就是离散状态的马尔可夫链。

看完了这个例子,应该对马尔科夫链有一定直观的了解了。那现在我们来讲讲什么是马尔科夫链。

\begin{quote}
考虑一个随机变量\(X=\{X_0,X_1,\cdots,X_t,\cdots\}\),表示下标t时刻的变量,每个随机变量\(X_t\)的集合取相同,成为状态空间\(\mathcal{S}\)。这个随机序列\(X\)构成\textbf{随机过程}。假设时刻\(t = 0\)时刻随机变量\(P(X_0)=\pi_0\),成为\textbf{初始分布}。假设在时刻\(t\geq1\)的随机变量\(X_t\)与\(X_{t+1}\)的分布\(P(X_t|X_{t-1})\),如果有:

\[P(X_t|X_0,X_1,\cdots ,X_{t-1})=P(X_t|X_{t-1}),\  \ t = 1,2, \cdots\]

即时刻t的状态只依赖于时刻t-1的状态,这个性质则成为\textbf{马尔科夫性},满足上面式子的随机序列\(X\)则成为\textbf{马尔可夫过程}或\textbf{马尔科夫链}。其中上述式子中\(P(X_t|X_{t-1})\)为\textbf{概率转移分布}
\end{quote}

马尔可夫性的直观解释是\textbf{下一时刻的状态只与现在的状态有关,而与过去的状态无关。}就像天气预报的例子,我们明天下雨的概率取决我们今天的天气状态。当然这个假设是比较粗糙的,但有时候也是合理的,比如我们的语义分析等。

如果马尔可夫链的转移概率分布是与时间无关的,也就是:

\[P_({X_{t+s}|X_{t-1+s}})=P({X_{t}|X}_{t-1}),\ \ \ t = 1,2,\cdots;\ \ s = 1,2,\cdots\]

这种马尔可夫链成为时间齐次的马尔可夫链,这篇文章提到的马尔可夫链都是其次的;同时本文考虑的均为一阶马尔可夫链。

\hypertarget{ux4e8cux79bbux6563ux72b6ux6001ux7684ux9a6cux5c14ux79d1ux592bux94fe}{%
\paragraph{二、离散状态的马尔科夫链}\label{ux4e8cux79bbux6563ux72b6ux6001ux7684ux9a6cux5c14ux79d1ux592bux94fe}}

\begin{enumerate}
\def\labelenumi{\arabic{enumi}.}
\item ~
  \hypertarget{ux8f6cux79fbux6982ux7387ux77e9ux9635ux548cux72b6ux6001ux5206ux5e03}{%
  \subparagraph{转移概率矩阵和状态分布}\label{ux8f6cux79fbux6982ux7387ux77e9ux9635ux548cux72b6ux6001ux5206ux5e03}}

  转移状态的马尔可夫链\(X=\{X_0,X_1,\cdots,X_t,\cdots\}\),随机变量\(X_t(t=0,1,2,\cdots)\)定义在离散空间\(\mathcal{S}\),设马尔可夫链在\((t-1)\)时刻状态j,在t时刻状态移动到i,这个转移概率可以写为:

  \[p_{ij} = (X_t=i|X_{t-1}=j),\ \ \ \ \ i=1,2,\cdots;\   j=i,2,\cdots\]

  满足:

  \[p_{ij}\geq0,\ \ \ \sum_{i}p_{ij}=1  \tag{2.1}\]

  因此我们的马尔可夫链转移矩阵:
\begin{equation}
  P=
  \begin{bmatrix} 
    p_{11} & p_{12} & p_{13} & \cdots\\
    p_{21} & p_{22} & p_{23} & \cdots\\
    p_{31} & p_{32} & p_{33} & \cdots\\
    \cdots & \cdots & \cdots & \cdots\\
    \end{bmatrix} 
\end{equation}

  马尔可夫链X在时刻t的概率分布为:
\begin{equation}
  \pi(t)=
  \begin{bmatrix} 
  \pi_{1}(t)\\
  \pi_{2}(t)\\
  \pi_{3}(t)\\
  \cdots\\
  \end{bmatrix} 
\end{equation}

  如果要计算下一时刻的状态,有:

  \[\pi(t)=P\pi(t-1) \tag{2.2}\]

  在这里我们解释下上面的一些方程。假设我们是从状态j转移到状态i,因此状态
  j
  (第二个下标)转移到下一时刻所有状态的概率之和为1,因此概率转移矩阵每一列求和的值为1。举个例子,我们今天是下雨天,那么明天是晴天的概率加上雨天的概率必然是1(前文假设天气只有晴雨两种状态)。如果我们今天晴天的概率为0.9,雨天的概率为0.1,那么明天的晴雨的概率则是两者的加权平均。容易证明下一时刻状态\(\pi(t)\)的所有状态和\(\pi_{1}+\pi_{2}+\pi_{3}+\cdots\)也为1:
\begin{equation}
  \pi(t)
  =
  \begin{bmatrix} 
  p_{11} & p_{12} & p_{13} & \cdots\\
  p_{21} & p_{22} & p_{23} & \cdots\\
  p_{31} & p_{32} & p_{33} & \cdots\\
  \cdots & \cdots & \cdots & \cdots\\
  \end{bmatrix} 
  \begin{bmatrix} 
  \pi_{1}(t-1)\\
  \pi_{2}(t-1)\\
  \pi_{3}(t-1)\\
  \cdots\\
  \end{bmatrix} 
  \\
  \left\{
      \begin{array}{l}
              \pi_{1}(t)=p_{11}\pi_{1}(t-1)+p_{12}\pi_{2}(t-1)+p_{13}\pi_{3}(t-1)+\cdots \\
              \pi_{2}(t)=p_{21}\pi_{1}(t-1)+p_{22}\pi_{2}(t-1)+p_{23}\pi_{3}(t-1)+\cdots \\
              \pi_{3}(t)=p_{31}\pi_{1}(t-1)+p_{32}\pi_{2}(t-1)+p_{33}\pi_{3}(t-1)+\cdots
      \end{array}
  \right.
\end{equation}


  之后我们对\(\pi\)累加:

  \[\sum_i{\pi_i(t)}=\pi_{1}(t-1)(p_{11}+p_{21}+p_{31}+\cdots)+
  		       \pi_{2}(t-1)(p_{12}+p_{22}+p_{32}+\cdots)+\cdots\]

  由于\(\sum_{i}p_{ij}=1\),因此可以得到:

  \[\sum_i{\pi_i(t)}=\sum_t{\pi_i(t-1)}\]

  设初始状态为\(\pi(0)\),容易得出:

  \[\sum_i{\pi_i(t)}=\sum_t{\pi_i(0)}=1\]

  同时:

  \[\pi(2) = P\pi(1)=P(P\pi(0))=P^2\pi(0)\]

  以此类推,可以得出:

  \[\pi(t) = P^{t}\pi(0)\]
\item ~
  \hypertarget{ux5e73ux7a33ux5206ux5e03}{%
  \subparagraph{平稳分布}\label{ux5e73ux7a33ux5206ux5e03}}

  \begin{quote}
  设有马尔科夫链\(X = \{X_0,X_1,\cdots,X_t,\cdots\}\),状态空间为\(\mathcal{S}\),概率转移矩阵为\(P=(p_{ij})\),如果存在一个分布\(\pi\)使得:

  \[\pi = P\pi\]

  则称为\(\pi\)马尔可夫链\(X = \{X_0,X_1,\cdots,X_t,\cdots\}\)的平稳分布

  平稳分布\(\pi = (\pi_1,\pi_2,\cdots)\)的充分必要条件是下述方程组的解:

  \[\left\{
  \begin{aligned}
  &x_i = \sum_{j}p_{ij}x_j \\
  &x_i\ge0,\ \ i=1,2,\cdots \\
  &\sum_{i}x_i=1
  \end{aligned}
  \right.\]
  \end{quote}
\end{enumerate}

\hypertarget{ux4e09ux8fdeux7eedux72b6ux6001ux7684ux9a6cux5c14ux79d1ux592bux94fe}{%
\paragraph{三、连续状态的马尔科夫链}\label{ux4e09ux8fdeux7eedux72b6ux6001ux7684ux9a6cux5c14ux79d1ux592bux94fe}}

理解了离散状态的马尔可夫链,那理解连续状态的马尔可夫链就会简单点了。假设连续状态的马尔可夫链为\(X=\{X_0,X_1,\cdots,X_t,\cdots\}\),随机变量为\(X_t(t=0,1,2,\cdots)\),在连续空间上的状态空间为\(\mathcal{S}\),转移概率分布由概率转移核或者转移核表示。离散状态的马尔可夫链是一个二维数组(矩阵),我们的概率转移核也是一个二维分布了。对于任意的\(x\in \mathcal{S}\),\(A\in \mathcal{S}\),转移核\(P(x,A)\)定义为:

\[P(x,A)=\int_A p(x,y)dy\]

其中\(p(x,\cdot)\)是概率密度函数,满足\(p(x,\cdot)\geq0\);

类似于公式\(2.1\),我们的转移矩阵的列向量,我们对整个状态空间积分:

\[P(x,\mathcal{S})=\int_{\mathcal{S}}p(x,y)dy=1\]

转移核\(P(x,A)\)表示\(x\sim A\)的转移概率:

\[P(X_t=A|X_{t-1}=x)=P(x,A)=\int_A p(x,y)dy\]

这个表示\(x\)状态下转移为\(A\)状态的概率。

其实可以对照下我们离散状态的概率分布。我们计算出来的下一步的概率密度其实是一个离散的分布。而连续状态的马尔可夫链算出来的是一个连续的分布。所以只能计算在\(A\in \mathcal{S}\)上的概率,或者是用一个概率分布来表示下一时刻的结果。

对于连续状态的马尔可夫链,若满足:

\[\pi (A)=\int P(x,A)\pi(x)dx\]

则为马尔可夫链的平稳分布,也可以写为:

\[\pi = P\pi\]
\subsection{马尔科夫链性质}

\hypertarget{ux4e00ux4e0dux53efux7ea6}{%
\paragraph{一、不可约}\label{ux4e00ux4e0dux53efux7ea6}}

\begin{quote}
设有马尔科夫链\(X = \{X_0,X_1,\cdots,X_t,\cdots\}\),状态空间为\(\mathcal{S}\),对于任意\(i,j \in \mathcal{S}\),如果存在一个时刻\(t(t>0)\),满足:

\[P(X_t = i|X_0 = j)>0\]

则成该马尔可夫链是不可约的,否则是可约的。即从时刻0出发,时刻t到达状态\(i\)的概率大于0,即一个不可约的马尔可夫链。
\end{quote}

直观地讲,在任意时刻出发,或者是任意状态出发,我们都可以到达状态\(i\),这就是不可约性。如果可约,从状态\(j\)出发,将会在一个或者多个状态下往返,而不会到达状态\(i\)。所以可约的马尔可夫链是没办法遍历所有的状态。

\hypertarget{ux4e8cux975eux5468ux671f}{%
\paragraph{二、非周期}\label{ux4e8cux975eux5468ux671f}}

\begin{quote}
设有马尔可夫链\(X=\{X_1,X_2,\cdots,X_t,\cdots\}\),状态空间是\(\mathcal{S}\),对于任意状态\(i\in \mathcal{S}\),如果时刻0从i出发,t时刻返回状态的所有时间长:\(\{t:P(X_t|X_0=i\}>0\)的最大公约数为1,则称为该马尔可夫链\(X\)是\textbf{非周期}的,否则则是周期的。
\end{quote}

周期性指的是,如果从马尔可夫链的一个状态\(i\)出发,经历了时间\(t\)之后回到状态\(i\),循环往复;而非周期性指的是我们从状态\(i\)出发,到状态\(i\)的时间不是固定的。如果又周期性,那如某个特定的事件段去观察该马尔科夫链,会发现我们只能观测到一个状态而已。

\begin{quote}
\textbf{定理:不可约,非周期的有限状态马尔可夫链有唯一平稳分布存在。}
\end{quote}

\hypertarget{ux4e09ux6b63ux5e38ux8fd4}{%
\paragraph{三、正常返}\label{ux4e09ux6b63ux5e38ux8fd4}}

\begin{quote}
有马尔可夫链\(X=\{X_1,X_2,\cdots,X_t,\cdots\}\),状态空间是\(\mathcal{S}\),对任意状态\(i,j\in \mathcal{S}\),定义概率\(p_{ij}^t\)为时刻0从状态\(j\)出发,时刻\(t\)为首次转移到状态\(i\)的概率,即:

\[p_{ij}^t=P(X_t=i,X_s\neq i,s=1,2,\cdots,t-1|X_0=j),t=1,2,\cdots\]

若\(\lim_{t\rightarrow\infty} p_{ij}^t>0\),则称为正常返的。
\end{quote}

我们若是一个正常返的马尔可夫链,从任意状态出发,当时间趋于无穷时,首次转移到状态\(i\)的概率不为0。

\begin{quote}
定理:不可约,非周期且正常反的马尔科夫链有唯一平稳分布的存在。
\end{quote}

\hypertarget{ux56dbux904dux5386ux5b9aux7406}{%
\paragraph{四、遍历定理}\label{ux56dbux904dux5386ux5b9aux7406}}

\begin{quote}
有马尔可夫链\(X=\{X_1,X_2,\cdots,X_t,\cdots\}\),状态空间是\(\mathcal{S}\),若马尔科夫链\(X\)是不可约、非周期且正常返的,则该马尔科夫链有唯一平稳分布\(\pi=(\pi_1,\pi_2,\cdots)^T\),并且转移概率的极限分布是马尔可夫链的平稳分布:

\[\lim_{t\rightarrow \infty}P(X_t=i|X_0=j)=\pi_i,\ i=1,2,\cdots;\ j=1,2,\cdots\]

若\(f(X)\)是定义在状态空间上的函数,且\(E_{\pi}[|f(X)|]<\infty\),则:

\[P\{ \hat{f_t}\rightarrow E_{\pi}[|f(X)|]\}=1\]

其中:

\[\hat{f_t}=\frac{1}{t}\sum_{s=1}^{t}f(x_s)\]

而\(E_{\pi}[f(X)]=\sum_if(i)\pi_i\)是\(f(X)\)关于平稳分布\(\pi\)的数学期望,即:

\[\hat{f_t} \rightarrow E_{\pi}[f(X)] ,\ \ \ t \rightarrow\infty\]

几乎处处成立(或者以概率1成立)。
\end{quote}

也就是意味着:满足遍历定理的马尔可夫链,当时间趋近于无穷时,马尔可夫链状态分布趋近于平稳分布,随机变量的函数的样本均值以概率1收敛于该函数的数学期望。这里我们可以认为:样本均值是时间均值,数学期望是空间均值。而遍历定理说明了当时间趋于无穷时,时间均值等于空间均值;而遍历定理的三个条件:不可约,非周期,正常返,让马尔可夫链在时间趋于无穷时到达任意状态的概率不为0。

这里需要理解的是时间均值和空间均值。\(\hat{f_t}\)可以理解为个体在该马尔可夫链上的游走的均值;而数学期望则是在平稳分布后的\(f(X)\)关于平稳分布\(\pi\)的期望,也就是说个体很难抵抗总体内禀的趋势。正如一位长者说过,\textbf{一个人的命运,当然要靠自我奋斗,但也要考虑到历史的进程。}

由于我们不清楚要多少次迭代马尔可夫链才能接近平稳分布,因此采用遍历定理时可以取一个足够大的m,当经过m次迭代之后的分布就认为是平稳分布(也就是所谓的burn-in),这时计算m+1次迭代到n次迭代的均值:

\[\hat{E}f=\frac{1}{n-m}\sum_{i=m+1}^nf(x_i)\]

这个期望称为遍历均值。

\hypertarget{ux4e94ux53efux9006ux9a6cux5c14ux79d1ux592bux94feux548cux7ec6ux81f4ux5e73ux8861ux65b9ux7a0b}{%
\paragraph{五、可逆马尔科夫链和细致平衡方程}\label{ux4e94ux53efux9006ux9a6cux5c14ux79d1ux592bux94feux548cux7ec6ux81f4ux5e73ux8861ux65b9ux7a0b}}

\begin{quote}
设有马尔可夫链\(X=\{X_1,X_2,\cdots,X_t,\cdots\}\),状态空间是\(\mathcal{S}\),转移概率矩阵为\(P\),如果状态分布\(\pi=(\pi_1,\pi_2,\cdots)^T\),对于任意状态\(i,j\in \mathcal{S}\),对于任意时刻\(t\)满足:

\[P(X_t=i|X_{t-1}=j)\pi_j=P(X_{t-1}=j|X_{t}=i)\pi_i,\ \ imj=1,2,\cdots\]

简写为:

\[p_{ji}\pi_{j}=p_{ij}\pi_i\]

则称此马尔可夫链是可逆马尔可夫链,上式\(p_{ji}\pi_{j}=p_{ij}\pi_i\)为细致平衡方程。

满足细致平衡方程的状态分布\(\pi\)就是该马尔可夫链的平稳分布,因此细致平衡方程也可以写为:

\[P\pi=\pi\]
\end{quote}
