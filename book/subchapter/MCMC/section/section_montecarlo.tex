\section{蒙特卡罗法 Monte Carlo Method}

\subsection{随机采样和接受-拒绝采样}

蒙特卡罗法是通过概率模型的随机抽样进行随机抽样的方法。假设概率分布已知,通过概率分布得到随机样本,并通过得到的随机样本得到概率分布的随机性质,因此蒙特卡洛方法的核心是随机抽样。接下来我们介绍\textbf{接受-拒绝采样}。

已知概率密度分布为\(f(x)\),但是这个概率密度分布复杂,各个变量并不独立,无法直接采样或者积分,因此可以通过蒙特卡罗方法进行抽样,得到样本\emph{\(X\)},得到其随机分布。我们在这介绍\textbf{接受-拒绝采样}。我们需要一个辅助的\textbf{建议分布},记为\(q(x)\)。这个建议分布可以产生我们的候选样本,但建议分布要满足:

\[c * q(x) \geq f(x)\]

之后我们对样本按照建议分布\(q(x)\)进行抽样,得到样本\(x^*\),同时对均匀分布\(U(0,1)\)进行抽样,得到\(u\)。之后计算\(\frac{f(x)}{c*q(x)}\)(这个值一定在0\textasciitilde1之间,对应图1的绿色部分比例),若

\[u \leq \frac{f(x^*)}{c*q(x^*)}\]

则\(x^*\)接受作为样本,否则拒绝。

怎么理解这个过程呢?简单来说就是我们先对建议分布\(q(x)\)的概率密度进行采样,因为这个比我们的\(f(x)\)更容易采样。假如这个分布很复杂,维度很高,直接算的话浪费计算资源,因此要先用一个简单的建议分布\(q(x)\)进行采样,得到建议的采样,但是这建议分布的采样终究不是我们需要的采样,所以我们需要在利用均匀分布\(U(0,1)\),由我们算出来的\(f(x^*)/cq(x^*)\)进行接受或者拒绝。在图1中,按照绿色比例进行接受。如果在第\(x^*_i\)个抽样刚好落在中间红色区域比较大的点,那么拒绝的概率就高,反之绿色部分的比例更大,则我们接受的概率就越高。

\begin{figure}
\centering
%\includegraphics{C:/Users/astro_gh/Desktop/QQ图片20201119220859.jpg}
\caption{}
\end{figure}

听到这是不是有点迷糊了?别着急!我们看看图1,在是不是\(x^*\)处红色部分占比越大,与目标分布\(f(x)\)相差就越远了?所以我们在这里就必须剔除一些点了,不然远离我们的真实分布了!
这时候可能会有其他疑问了,那在图1两端概率很小时候岂不是都接受率很高?是的,但是那两端概率很低呀!因为我们在使用建议分布抽样的时候概率那里的点已经是很少了,所以我们不用拒绝很多样本点也就和目标分布类似了。
所以这时候就要用到一个均匀分布\(U(0,1)\),在该点上随机生成一个\(u\),然后按照\((1.1)\)则接受,否则拒绝。

所以假设我们抽了n个样本,对样本进行拒绝,就是要生成和判断n次\(u\)的取值,也就是对每个样本点进行计算和判断是否拒绝,完成一次拒绝-接受采样。

接受-拒绝采样的缺点也是有的,主要是接受率比较低,抽样效率低。
\subsection{数学期望和蒙特卡罗积分}
如果我们要算目标函数为\(f(x)\),其概率密度为\(p(x)\),我们记函数\(f(x)\)关于密度函数\(p(x)\)的数学期望为\(E_{p(x)}[f(x)]\)。我们按照密度函数\(f(x)\)独立抽取n个样本\(x_1,x_2\cdots,x_n\),之后计算样本的均值:

\[\hat{f}_n=\frac{1}{n}\sum_{i=1}^{n}f(x_i)\]

作为\(f(x)\)的近似值。根据大数定理,当样本容量增大,样本均值以概率1收敛于数学期望。因此我们可以用上述方法得到我们的数学期望。

\[E_{p(x)}[f(x)]\approx\frac{1}{n}\sum_{i=1}^{n}f(x_i)\]

而在\(\mathcal{X}\)上数学期望的积分形式为:

\[E(x)=\int_{\mathcal{X}} f(x)p(x)dx\]

如果我们的目标积分为:

\[h(x)=f(x)q(x)\]

我们可以写为:

\[\int_{\mathcal{X}}h(x)dx=\int_{\mathcal{X}}f(x)q(x)dx=E_{p(x)}[f(x)]\]

对于复杂的函数,可以给定一个概率密度函数\(p(x)\),只要取

\[f(x)=\frac{h(x)}{p(x)}\]

就可以用\(p(x)\)进行抽样算出积分:

\[\int_{\mathcal{X}}h(x)dx=E_{p(x)}[f(x)]\approx\frac{1}{n}\sum_{i=1}^{n}f(x_i)\]

因此我们可以先用\(p(x)\)抽样。然后再进行积分。
