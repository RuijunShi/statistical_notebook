\section{MCMC原理}

\subsection{MCMC原理}

我们简单介绍了蒙特卡罗方法和马尔可夫链,接下来我们介绍马尔可夫链蒙特卡罗方法,下面简称MCMC方法。MCMC方法适用于随机变量多元的、密度函数是非标准形式的、随机变量不相互独立的情况。若存在维数太高的情况,直接抽样是不可能的,因为其数据量在指数级增长,MCMC就可以避免维数爆炸的问题。

假设多元随机变量\(x = [x_{1},x_{2},x_{3},\cdots]\),满足\(x\in \mathcal{X}\)且其概率密度为\(p(x)\),\(f(x)\)是定义在\(x\in \mathcal{X}\)上的函数,我们的目标是获得概率分布\(p(x)\)的抽样以及\(f(x)\)的数学期望\(E_{p(x)}[f(x)]\)。在随机变量x的状态空间\(\mathcal{S}\)上满意遍历定理(上一章2.2马尔科夫链性质)的马尔可夫链\(X = \{X_0,X_1,\cdots,X_t,\cdots\}\),当这个马尔可夫链平稳时的分布就是其抽样的目标分布\(p(x)\)。

怎么通俗地解释这个原理呢?首先构造一个马尔可夫链,在状态空间\(\mathcal{S}\)上进行随机游走。根据遍历原理,总有一个时刻m之后,这个马尔可夫链近于平稳分布,也就是在期望附近游走。假设我们随机游走了n步,取我们平稳分布之后的样本集合\(\{x_{m+1},x_{m+2},x_{m+3},\cdots,x_{n}\}\),就是我们目标抽样分布的结果。

这里又要唠叨一句,平稳分布只是各个状态的期望。我们用天气由于预报的例子来说,假设当天气平稳分布的时候,其平稳分布为\([0.7 \ 0.3]^T\),我们之后观测天气的结果符合我们这个平稳分布,也就是说我们接下来100天中有70天是晴天,30天是雨天。当然这只是一个简单的假设模型罢了!

所以当到达平稳之后样本集合为\(\{x_{m+1},x_{m+2},x_{m+3},\cdots,x_{n}\}\)就是我们目标分布的结果。在时刻m之前的时期我们称为\textbf{燃烧期} (burn-in)。

更晕的还在后边,我们怎么构造这样一个马尔可夫链?这里我们需要一个转移核(连续)或者转移矩阵(离散)。如何构造这转移核/矩阵,构成一个可逆的马尔可夫链,使得遍历定理成立是很关键的。如果该马尔可夫链成立,由于遍历定理成立,因此初始值的选取最终会收敛到同一平稳分布;燃烧期之前的样本都要丢弃,因为燃烧期之前的样本都不是服从样本的分布。当然目前MCMC收敛的判断是经验性的。

MCMC方法比拒绝-接受采样更容易实现,虽然丢弃了燃烧器之前的样本,但其效率仍然比拒绝-接受采样的效率高。目前常用的MCMC
方法主要是Metropolis-Hasting算法(M-H算法)和吉布斯抽样。

\hypertarget{32-mcmcux7b97ux6cd5}{%
\subsection{MCMC算法}\label{32-mcmcux7b97ux6cd5}}

根据上面的介绍,MCMC方法可以是以下的步骤:

\begin{quote}
\begin{enumerate}
\def\labelenumi{\arabic{enumi}.}
\item 在随机变量\(x\)的状态空间\(\mathcal{S}\)上构造一个满足遍历定理的马尔可夫链,使得其平稳分布为\(p(x)\);
\item 在状态空间某一点\(x_0\)出发,构造随机游走,产生样本\(x_0,x_1,x_2,\cdots,x_t,\cdots\);
\item 应用遍历定理,确定燃烧期m,求得函数\(f(x)\)的均值
  \[\hat{E}f=\frac{1}{n-m}\sum_{i=m+1}^{n}f(x_i)\]
\end{enumerate}
\end{quote}
在这有几个问题:
\begin{quote}
\begin{enumerate}
\def\labelenumi{\arabic{enumi}.}
\item 如何定义马尔可夫链
\item 如何确定收敛步骤
\item 如何确定迭代步数确保精度
\end{enumerate}
\end{quote}
