\chapter{MCMC与采样方法}
\section{蒙特卡罗法 Monte Carlo Method}

\subsection{随机采样和接受-拒绝采样}

蒙特卡罗法是通过概率模型的随机抽样进行随机抽样的方法。假设概率分布已知,通过概率分布得到随机样本,并通过得到的随机样本得到概率分布的随机性质,因此蒙特卡洛方法的核心是随机抽样。接下来我们介绍\textbf{接受-拒绝采样}。

已知概率密度分布为\(f(x)\),但是这个概率密度分布复杂,各个变量并不独立,无法直接采样或者积分,因此可以通过蒙特卡罗方法进行抽样,得到样本\emph{\(X\)},得到其随机分布。我们在这介绍\textbf{接受-拒绝采样}。我们需要一个辅助的\textbf{建议分布},记为\(q(x)\)。这个建议分布可以产生我们的候选样本,但建议分布要满足:

\[c * q(x) \geq f(x)\]

之后我们对样本按照建议分布\(q(x)\)进行抽样,得到样本\(x^*\),同时对均匀分布\(U(0,1)\)进行抽样,得到\(u\)。之后计算\(\frac{f(x)}{c*q(x)}\)(这个值一定在0\textasciitilde1之间,对应图1的绿色部分比例),若

\[u \leq \frac{f(x^*)}{c*q(x^*)}\]

则\(x^*\)接受作为样本,否则拒绝。

怎么理解这个过程呢?简单来说就是我们先对建议分布\(q(x)\)的概率密度进行采样,因为这个比我们的\(f(x)\)更容易采样。假如这个分布很复杂,维度很高,直接算的话浪费计算资源,因此要先用一个简单的建议分布\(q(x)\)进行采样,得到建议的采样,但是这建议分布的采样终究不是我们需要的采样,所以我们需要在利用均匀分布\(U(0,1)\),由我们算出来的\(f(x^*)/cq(x^*)\)进行接受或者拒绝。在图1中,按照绿色比例进行接受。如果在第\(x^*_i\)个抽样刚好落在中间红色区域比较大的点,那么拒绝的概率就高,反之绿色部分的比例更大,则我们接受的概率就越高。

\begin{figure}
\centering
%\includegraphics{C:/Users/astro_gh/Desktop/QQ图片20201119220859.jpg}
\caption{}
\end{figure}

听到这是不是有点迷糊了?别着急!我们看看图1,在是不是\(x^*\)处红色部分占比越大,与目标分布\(f(x)\)相差就越远了?所以我们在这里就必须剔除一些点了,不然远离我们的真实分布了!
这时候可能会有其他疑问了,那在图1两端概率很小时候岂不是都接受率很高?是的,但是那两端概率很低呀!因为我们在使用建议分布抽样的时候概率那里的点已经是很少了,所以我们不用拒绝很多样本点也就和目标分布类似了。
所以这时候就要用到一个均匀分布\(U(0,1)\),在该点上随机生成一个\(u\),然后按照\((1.1)\)则接受,否则拒绝。

所以假设我们抽了n个样本,对样本进行拒绝,就是要生成和判断n次\(u\)的取值,也就是对每个样本点进行计算和判断是否拒绝,完成一次拒绝-接受采样。

接受-拒绝采样的缺点也是有的,主要是接受率比较低,抽样效率低。
\subsection{数学期望和蒙特卡罗积分}
如果我们要算目标函数为\(f(x)\),其概率密度为\(p(x)\),我们记函数\(f(x)\)关于密度函数\(p(x)\)的数学期望为\(E_{p(x)}[f(x)]\)。我们按照密度函数\(f(x)\)独立抽取n个样本\(x_1,x_2\cdots,x_n\),之后计算样本的均值:

\[\hat{f}_n=\frac{1}{n}\sum_{i=1}^{n}f(x_i)\]

作为\(f(x)\)的近似值。根据大数定理,当样本容量增大,样本均值以概率1收敛于数学期望。因此我们可以用上述方法得到我们的数学期望。

\[E_{p(x)}[f(x)]\approx\frac{1}{n}\sum_{i=1}^{n}f(x_i)\]

而在\(\mathcal{X}\)上数学期望的积分形式为:

\[E(x)=\int_{\mathcal{X}} f(x)p(x)dx\]

如果我们的目标积分为:

\[h(x)=f(x)q(x)\]

我们可以写为:

\[\int_{\mathcal{X}}h(x)dx=\int_{\mathcal{X}}f(x)q(x)dx=E_{p(x)}[f(x)]\]

对于复杂的函数,可以给定一个概率密度函数\(p(x)\),只要取

\[f(x)=\frac{h(x)}{p(x)}\]

就可以用\(p(x)\)进行抽样算出积分:

\[\int_{\mathcal{X}}h(x)dx=E_{p(x)}[f(x)]\approx\frac{1}{n}\sum_{i=1}^{n}f(x_i)\]

因此我们可以先用\(p(x)\)抽样。然后再进行积分。

\section{MCMC原理}

\subsection{MCMC原理}

我们简单介绍了蒙特卡罗方法和马尔可夫链,接下来我们介绍马尔可夫链蒙特卡罗方法,下面简称MCMC方法。MCMC方法适用于随机变量多元的、密度函数是非标准形式的、随机变量不相互独立的情况。若存在维数太高的情况,直接抽样是不可能的,因为其数据量在指数级增长,MCMC就可以避免维数爆炸的问题。

假设多元随机变量\(x = [x_{1},x_{2},x_{3},\cdots]\),满足\(x\in \mathcal{X}\)且其概率密度为\(p(x)\),\(f(x)\)是定义在\(x\in \mathcal{X}\)上的函数,我们的目标是获得概率分布\(p(x)\)的抽样以及\(f(x)\)的数学期望\(E_{p(x)}[f(x)]\)。在随机变量x的状态空间\(\mathcal{S}\)上满意遍历定理(上一章2.2马尔科夫链性质)的马尔可夫链\(X = \{X_0,X_1,\cdots,X_t,\cdots\}\),当这个马尔可夫链平稳时的分布就是其抽样的目标分布\(p(x)\)。

怎么通俗地解释这个原理呢?首先构造一个马尔可夫链,在状态空间\(\mathcal{S}\)上进行随机游走。根据遍历原理,总有一个时刻m之后,这个马尔可夫链近于平稳分布,也就是在期望附近游走。假设我们随机游走了n步,取我们平稳分布之后的样本集合\(\{x_{m+1},x_{m+2},x_{m+3},\cdots,x_{n}\}\),就是我们目标抽样分布的结果。

这里又要唠叨一句,平稳分布只是各个状态的期望。我们用天气由于预报的例子来说,假设当天气平稳分布的时候,其平稳分布为\([0.7 \ 0.3]^T\),我们之后观测天气的结果符合我们这个平稳分布,也就是说我们接下来100天中有70天是晴天,30天是雨天。当然这只是一个简单的假设模型罢了!

所以当到达平稳之后样本集合为\(\{x_{m+1},x_{m+2},x_{m+3},\cdots,x_{n}\}\)就是我们目标分布的结果。在时刻m之前的时期我们称为\textbf{燃烧期} (burn-in)。

更晕的还在后边,我们怎么构造这样一个马尔可夫链?这里我们需要一个转移核(连续)或者转移矩阵(离散)。如何构造这转移核/矩阵,构成一个可逆的马尔可夫链,使得遍历定理成立是很关键的。如果该马尔可夫链成立,由于遍历定理成立,因此初始值的选取最终会收敛到同一平稳分布;燃烧期之前的样本都要丢弃,因为燃烧期之前的样本都不是服从样本的分布。当然目前MCMC收敛的判断是经验性的。

MCMC方法比拒绝-接受采样更容易实现,虽然丢弃了燃烧器之前的样本,但其效率仍然比拒绝-接受采样的效率高。目前常用的MCMC
方法主要是Metropolis-Hasting算法(M-H算法)和吉布斯抽样。

\hypertarget{32-mcmcux7b97ux6cd5}{%
\subsection{MCMC算法}\label{32-mcmcux7b97ux6cd5}}

根据上面的介绍,MCMC方法可以是以下的步骤:

\begin{quote}
\begin{enumerate}
\def\labelenumi{\arabic{enumi}.}
\item 在随机变量\(x\)的状态空间\(\mathcal{S}\)上构造一个满足遍历定理的马尔可夫链,使得其平稳分布为\(p(x)\);
\item 在状态空间某一点\(x_0\)出发,构造随机游走,产生样本\(x_0,x_1,x_2,\cdots,x_t,\cdots\);
\item 应用遍历定理,确定燃烧期m,求得函数\(f(x)\)的均值
  \[\hat{E}f=\frac{1}{n-m}\sum_{i=m+1}^{n}f(x_i)\]
\end{enumerate}
\end{quote}
在这有几个问题:
\begin{quote}
\begin{enumerate}
\def\labelenumi{\arabic{enumi}.}
\item 如何定义马尔可夫链
\item 如何确定收敛步骤
\item 如何确定迭代步数确保精度
\end{enumerate}
\end{quote}

\section{Metropolis-Hastings采样}

\subsection{M-H采样原理}

上一章我们讲了MCMC抽样的一些问题,这一章我们介绍MCMC的一种代表算法。这一小节我们介绍M-H采样的原理。

我们需要构造一条马尔可夫链;要构造一个转移核,使得平稳分布就是我们要的抽样分布。参考第二章2.2节的细致平衡:

\[p_{ji}\pi_{j}=p_{ij}\pi_{i}\]

当然要构造这个细致平衡条件很难。假设\textbf{目标分布}为\(\pi(x)\),转移核为\(q(i,j)\),通常我们只能得到这样的结果:

\[\pi(i)q(i,j) \neq \pi(j) q(j,i)\]

因此我们要构造一个分布能使得其分布是细致平衡。

假设我们通过建议分布\(q(i,j)\)中随机抽取一个\textbf{候选状态\(x_j\)(后面简写为\(j\))},我们可以在两端乘以一个\(\alpha(i,j)\)使得其变为平稳分布,即:

$$\pi(i)q(i,j)\alpha(i,j) = \pi(j)q(j,i)\alpha(j,i)$$

在这里\(\alpha(i,j)\)为\textbf{接受分布}。建议分布\(q(i,j)\)是马尔可夫链转移核,且该马尔可夫链是不可约的,同时这个分布是容易采样的。那这个接受分布怎么构建呢?

我们对\( \pi(i)q(i,j)\alpha(i,j) = \pi(j)q(j,i)\alpha(j,i) \) 移项:

\[\alpha(i,j) =\frac{\pi(j)(j,i)}{\pi(i)q(i,j)}\alpha(j,i)\]

实际上我们需要把两边的接受分布扩大到1,这样接受率才会达到最大。如果\(\alpha(j,i)=1\),有:

\[\alpha(i,j) =\frac{\pi(j)q(j,i)}{\pi(i)q(i,j)}\]

相反的,如果上式\(\alpha(i,j)>1\),我们令\(\alpha(i,j)=1\):

\[1=\frac{\pi(j)q(j,i)}{\pi(i)q(i,j)}\alpha(j,i)\]

即我们取\(\alpha(i,j)=1\)。这时候接受分布\(\alpha\)扩大到最大。因此得到接受分布:

\[\alpha(i,j)=\min \left\{1,\frac{\pi(j)q(j,i)}{\pi(i)q(i,j)} \right\}\]

这时候的接受率最高,且容易证明这个转移核\(q(i,j)\alpha(i,j)\)是平稳分布的。之后我们从区间\((0,1)\)中均匀采样,得到一个随机数\(u\),按照以下判断:

\[x_t=
\left\{
\begin{aligned}
x_j,\ \ u\leq\alpha(i,j) \\
x_i, \ \ u  > \alpha(i,j)
\end{aligned}
\right.\]

决定其是否接受下一步。

怎么理解这个接受分布呢?前面提到,当我们从状态\(i\)以概率\(p(i,j)\)转移到状态\(j\)的时候,不一定是平稳的,而加入\(\alpha\)之后就可以得到一个新的平稳的马尔可夫链。而我们是否要转移到下一步就是要考虑这个接受分布了。我们定性的解释这个问题。假设我们的建议分布\(q(i,j)=q(j,i)\),我们的接受分布\(\alpha=\min\{1,\frac{\pi_j}{\pi_i}\}\),如果\(\alpha\)太小,说明我们下一步抽取的\(j\)所对应的概率是远小于我们上一步抽取的概率\(i\),所以要舍弃;而如果是1的话说明抽取的\(j\)很符合我们的目标分布\(\pi\),因此可以更好地接近我们要抽样的分布。这个判断步骤类似于本文开头的拒绝-接受采样。\(\alpha\)类似于接受-拒绝采样的\(\frac{f(x)}{c*q(x)}\)。综上,其实这个转移核是要根据我们的抽样\(i,j\)共同确定的。
\subsection{M-H采样算法}

\begin{figure}
\centering
%\includegraphics{http://cos.name/wp-content/uploads/2013/01/mcmc-algo-2.jpg}
\caption{}
\end{figure}
常见的采样有ptmcmc\cite{justin_ellis_2017_1037579}

\section{吉布斯采样}

\subsection{满条件分布}

MCMC要抽样的函数一般都是多变量的联合概率分布\(p(x)=p(x_1,x_2,\cdots,x_k)\),其中\(x=(x_1,x_2,\cdots,x_k)^T\)是\(k\)维变量。若条件概率分布:\(p(x_I|x_{-I})\)中出现了所有的变量\(k\),其中:

\[x_I=\{x_{i},i\in I\},\ x_{-I}=\{x_{i},i\notin I\}\ \ \  \ \ \  \ I\subset K=\{1,2,\cdots,k\}\]

那么称这个分布为\textbf{满条件分布}。
\subsection{Gibbs采样原理}

当然M-H采样有接受分布的存在,因此效率还是不够高。Gibbs采样可以避免这个问题。吉布斯采样的基本原理是从满条件概率分布出发,从满条件概率分布中抽样,得到一个样本序列。基本原理是:吉布斯抽样过程是在一个马尔可夫链上随机游走,平稳分布就是目标联合分布。接下来我们介绍Gibbs采样的细节。

我们考虑二维情况:假设有一个二维分布\(p(x,y)\),我们发现:

\[p(x_1,y_1)p(y_2|x_1)=p(x_1)p(y_1|x_1)p(y_2|x_1) \\
p(x_1,y_2)p(y_1|x_1)=p(x_1)p(y_2|x_1)p(y_1|x_1)\]

整理得:

\[p(x_1,y_1)p(y_2|x_1)=p(x_1,y_2)p(y_1|x_1)\]

假设点\(A\)为\((x_1,y_1)\),点\(B\)为\((x_1,y_2)\),我们可以改写为:

\[p(A)p(y_2|x_1)=p(B)p(y_1|x_1)\]

也就是说当点A转移到点B的时候服从上述的马尔可夫链。而上式的条件概率就是我们的转移矩阵或者转移核。也就是说对维度\(y\)的满条件分布即为上述马尔可夫链的转移核或者转移矩阵。如果把二维扩展到\(n\)维,即可以得到\(n\)维的吉布斯抽样。假设建议分布是当前变量\(x_j,\ j=1,2,\cdots,k\)(也就是抽取第\(j\)维变量)的满条件分布:\(q(x',x)=p(x'_j|x_{-j})\),这里的\(x\)指的是当前的抽样,\(x'\)指的是下一步的抽样。扩展到维的情多维:

\[p(x'_j,x_{-j})p(x'_j|x_{-j})=p(x_j,x_{-j})p(x_j|x_{-j})\]

这时候的接受率\(\alpha=1\)。因此吉布斯采样可以认识是M-H采样的一种特殊情况。这个建议分布就是我们的目标分布的满条件分布。(这里和上一章的符号有所变化,不过内容是一样的,只是为了方便表达。)下面证明如何得到接受率\(\alpha=1\):

根据M-H采样的接受分布公式,有:

\[q(x,x')=p(x'_j|x_{-j})\]

代入接受分布:

\[\alpha(x,x')
=min\left\{ 1,\frac{p(x')q(x',x)}{p(x)q(x,x')}  \right\}\]

因为我们是抽取当前变量\(j\),因此有:
\begin{equation}
  \begin{aligned}
    \alpha(x,x')
    &=min\left\{ 1,\frac{p(x')q(x',x)}{p(x)q(x,x')}  \right\} \\
    &=min\left\{ 1,
    \frac{p(x'_j,x_{-j})p(x'_j|x_{-j})}
    {p(x_j,x_{-j})p(x_j|x_{-j})}  \right\}=1
    \end{aligned}
\end{equation}


之后抽取\(k\)维,循环\(n\)次,抛去燃烧期\(m\),得到我们的抽样,计算均值。
\subsection{Gibbs采样算法}

\begin{figure}
\centering
%\includegraphics{http://cos.name/wp-content/uploads/2013/01/gibbs-algo-2.jpg}
\caption{}
\end{figure}


\section{Nested采样}
对于evidence来说要进行高维积分,这是一个非常大的开销,因此还有另外的比较常用的MCMC积分:MultiNest 算法。一个模型的参数空间的活动的点先填充先验;

\section{Savage-Dickey density ratio}
我们发现Bayes Factor的计算比较困难,需要多重积分,因此使用\textbf{Savage-Dickey density ratio}的方式估计Bayes Factor。

我们考虑一个包含假设$\mathcal{H}_1$和假设$\mathcal{H}_2$的嵌套的模型,这个模型有共同参数$\theta$。假设$\mathcal{H}_1$有自己特有的参数A,可以通过设置$A=0$得到假设$\mathcal{H}_2$:
\begin{equation}
    p(d|A=0,\theta;\mathcal{H}_1)=p(d|\theta;\mathcal{H}_2)
\end{equation}
当A=0时,后验密度为:
\begin{equation}
    \begin{aligned}
        p(A=0|d;\mathcal{H}_1)&=\int p(A=0,\theta|d;\mathcal{H}_1)\mathrm{d}^n\theta\\
        &=\int \frac{p(d|A=0,\theta;\mathcal{H}_1)p(A=0)p(\theta)}{p(d|\mathcal{H}_1)}\mathrm{d}^n\theta\\
        &=\int \frac{p(d|\theta;\mathcal{H}_2)p(A=0)p(\theta)}{p(d|\mathcal{H}_1)}\mathrm{d}^n\theta\\
        &=\frac{p(A=0)}{p(d|\mathcal{H}_1)}\int p(d|\theta;\mathcal{H}_2)p(\theta)\mathrm{d}^n\theta\\
        &=\frac{p(d|\mathcal{H}_2)}{p(d|\mathcal{H}_1)}p(A=0)
    \end{aligned}
\end{equation}
因此Bayes Factor为:
\begin{equation}
    \mathcal{B}_{12}=\frac{p(d|\mathcal{H}_1)}{p(d|\mathcal{H}_2)}=\frac{p(A=0)}{p(A=0|d;\mathcal{H}_1)}
\end{equation}
相应的就是A=0时的先验和归一化后验之比。在写程序的时候$A=0$会导致许多的错误,因此一般会把A设置成一个非常小的数,比如在pulsar timing的探测中,会使用$\log_{10}A_{\mathrm{low}}=-18$,这个振幅远远低于pulsar的固有噪声。

\section{Product space sampling}
很多时候我们给出的后验采样是很难采样到A=0时的后验概率分布,因此需要\textbf{Product space sampling}。原理是通过增加一个超参数$n$进行odds ratio的计算。我们假设有n个待选的模型$H_i$,其中$i\in \{1,\cdots,n\}$。我们把Bayes定理写为:
\begin{equation}
    p(\theta|d,H_n)=\frac{p(d|\theta,H_n)p(\theta|H_n)}{p(d|H_n)}\equiv \frac{\mathcal{L}\pi}{\mathcal{Z}}
\end{equation}
其中$\mathcal{L}$为likelihood function,$\pi$为先验分布,$\mathcal{Z}$为的evidence,可以写为:
\begin{equation}
    \mathcal{Z}=\int_{\textrm{all }\theta}\mathcal{L}(\theta)\pi(\theta)\textrm{d}\theta
\end{equation}
对于我们的假设(或者说模型)$H_i,i\in{1,\cdots,n}$的后验概率为:
\begin{equation}
    \begin{aligned}
        p(H_i|d)&=\frac{p(d|H_i)p(H_i)}{p(d)}\\
        &=\frac{\int_{\theta_i} p(d|\theta_i,H_i)p(\theta_i)\textrm{d}\theta_i p(H_i)}{p(d)}\\
        &=\frac{\mathcal{Z}_i\pi_{H_i}}{p(d)}
    \end{aligned}
\end{equation}
根据OR的定义[\ref{ORs}],我们的后验ORs可以写为:
\begin{equation}
    \ln \mathcal{O}_{ji}=\mathcal{P}_{ji}
    =\ln\left[ \frac{p(H_j|d)}{p(H_i|d)} \right]
    =\ln\left(\frac{\mathcal{Z}_j}{\mathcal{Z}_i}\right)+\ln \left(\frac{\pi_{H_j}}{\pi_{H_i}}\right)
\end{equation}
当$\pi_{H_i}=\pi_{H_j}$时,BF=OR。注意这里的ORs和前面定义的ORs是互为倒数的。

这样计算ORs或者BF的时候要计算evidence,但这个计算成本太高了,我们可以试着转换为我们熟悉的MCMC采样。我们的假设模型$H_n(n\in \{1,\cdots,N\})$可以考虑合并为一个hypermodel。对于模型$H_n$来说,下表n就像一个开关。而各个模型的参数$\theta_n$可以合并为一个参数向量$\theta$。对于模型选择参数n可以连续化,再整数化转为整数参数n。

一般的,有两个不同的模型假设,其参数向量:$\theta_n$和$\theta_{n'}$,一般这两个参数的维度不一样。如果有$\theta_n \subset \theta_{n+1}$这种情况,一般会考虑nested model,并使用reversible-jump Markov chain Monte Carlo (RJMCMC)实现不同维度空间的转换。这里介绍另外的方法。我们需要知道假设N的先验是已知的,对参数空间$\theta$加入一个新的参数n,得到整体参数空间$(\theta,n)$,这个参数空间的先验已经确定下来了。这样我们就可以使用传统的MCMC采样方法或者是nested sampling。对于任意给定的参数n,联合参数空间可以划分为假设$H_n$的参数$\theta_n$和不属于其假设的参数$\phi_n$。也就是说:
\begin{equation}
    p(d|n,\theta)=p(d|n,\theta)
\end{equation}
在我们规定的hypermodel $\mathcal{H}$下会将参数$\theta_n$传递到假设$H_n$;而剩下的参数$\phi_n$将会被忽略,在参数空间上赋予一个常数值。如果参数空间维度比较大的情况下,很可能出现参数重叠的情况,这里我们考虑nested 模型。

当参数空间$(\theta,n)$获得到一组后验分布时,我们可以计算归一化后验分布:
\begin{equation}
    p(n|d,\mathcal{H})=\int p(\theta,n|d,\mathcal{H})\textrm{d}\theta
    =\frac{1}{\mathcal{Z}_{\mathcal{H}}}\int\mathcal{L}(\theta,n)\pi(\theta,n)\textrm{d}\theta
\end{equation}
$\mathcal{Z}_{\mathcal{H}}$是Hypermodel的evidence。先验可以写为:
\begin{equation}
    \pi(\theta|n)=\pi(\theta_n|n)\pi(\phi_n)\pi(n)
\end{equation}
为了让推导更加清晰,在这里展开得到关于参数空间$(\theta,n)$的后验分布:
\begin{equation}
    p(n,\theta|d)=\frac{p(d|n,\theta_n)p(\theta_n|n)p(\phi_n|\theta_n,n)p(n)}{p(d)}
\end{equation}
实际上分子第一项对应的是关于模型n的likelihood function;而分母项则是hypermodel的evidence $\mathcal{Z}_{\mathcal{H}}$。对参数$\theta$积分得到:
\begin{equation}
    p(n|d,\mathcal{H})=\frac{\pi(n)}{\mathcal{Z}_{\mathcal{H}}}\int\mathcal{L}(\theta_n)\pi(\theta_n|n)\textrm{d}\theta  \label{post_n}
\end{equation}
在这里我们利用先验的归一化:$\int \textrm{d}\phi_n=1$。我们注意到\ref{post_n}是正是模型$\mathcal{H}_n$的evidence。因此我们可以得到:
\begin{equation}
    \pi(n)\mathcal{Z}_n=\mathcal{Z}_{\mathcal{H}}p(n|d,\mathcal{H})
\end{equation}
因此ORs可以写为:
\begin{equation}
    \ln \mathcal{O}_{ji} = \ln\left[\frac{p(n=j|d,\mathcal{H})}{p(n=i|d,\mathcal{H})}\right]
\end{equation}
我们发现关于hypermodel的evidence就被约掉了,因此避免了繁杂的evidence的计算。
这个方法的核心算法\cite{godsill_relationship_2001,hee_bayesian_2016}:
\begin{equation}
    \theta_i \sim p(\theta_i|\phi_i,n,d)\propto 
    \begin{cases}
        p(y|n,\theta_n)p(\theta_n|n),i=n\\
        p(\theta_n|\phi_n,n),i\neq n,\\
    \end{cases}
\end{equation}
\begin{equation}
    k\sim p(n|\theta,d)\propto p(d|n,\theta_n)p(\theta_n|n)p(\phi_n|\theta_n,n)p(n)
\end{equation}
当然在这里我们的evidence只是一个归一化常数,当我们计算ORs或者BF的时候,hypermodel的evidence会被除掉,因此可以不用计算evidence的积分,而采用传统的MCMC采样方法。